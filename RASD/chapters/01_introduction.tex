\chapter{Introduction}

\section{Purpose}
The purpose of this document is to present a detailed description of CodeKataBattle (CKB).
It provides functional and non-functional requirements for the development of the system, including use cases, features, user interaction and system constraints. This document is addressed to the developers who have to implement the requirements and
could be used as an agreement between the customer and the contractors.

CodeKataBattle (CKB) is a new platform that helps students improve their software development skills by training with peers on code kata. Educators use the platform to challenge students by creating code
kata battles in which teams of students can compete against each other, thus proving (and improving) their skills and soft-skills : in fact working in a team is useful to learn soft-skills such as
communication and coordination.

A code kata battle is essentially a programming exercise in a programming language of choice (e.g.,
Java, Python). The exercise includes a brief textual description and a software project with build
automation scripts (e.g., a Gradle project in case of Java sources) that contains a set of test cases that
the program must pass, but without the program implementation. Students are asked to complete the
project with their code. In particular, groups of students participating in a battle are expected to follow
a test-first approach and develop a solution that passes the required tests. Groups deliver their
solution to the platform (by the end of the battle). At the end of the battle, the platform assigns scores
to groups to create a competition rank.

\subsection{Goals}
% \par\noindent\rule{
% \begin{table}[h]
%     \begin{tabular}{ c p{2cm} }
%         \hline
%         \textbf{G1} & Allows Students to improve their software development skills by partecipating in coding tournaments where they write programs \\
%         \textbf{G2} & Allows Educators to challenge Students by create coding tournaments and battles.                                              \\
%         \hline
%     \end{tabular}
%     \label{tab:Goals}
% \end{table}
% }

\begin{description}
    \item \textbf{G1} \quad Allows Students to improve their software development skills by partecipating in coding tournaments where they write programs
    \item \textbf{G2} \quad Allows Educators to challenge Students by create coding tournaments and battles.
    \item \textbf{Gi} \quad Allows Educators to track Students' knowledge about software development, evaluating the code written during battles and the overall score obtained in tournaments
    \item \textbf{Gi} \quad Allows Educators to customize battles defining specific rules, achivements that Students can obtain and evaluation modality.
    \item \textbf{Gi} \quad Allows Students to improve their soft skills, as communication, collaboration and time management, by creating teams for coding battles
    \item {\textcolor{red}{Non son sicuro siano goal che Code kata vuole risolvere}}
    \item \(G_i\) Allows S to learn sw develoment divertendosi e confrontandosi con i compagni in sane competizioni con ranking
\end{description}
\par\noindent\rule{\textwidth}{0.5pt}

\blindtext
\section{Scope}

\subsection{World Phenomena}
\par\noindent\rule{\textwidth}{0.5pt}
\begin{description}
    \item \textbf{WP1} \quad The students fork the GitHub repository of the code kata
    \item \textbf{WP1} \quad Students set up an automatic workflow in the GitHub repository
    \item \textbf{WP1} \quad The students work on the code kata battle
    \item \textbf{WP1} \quad The students push their work to the GitHub repository
    \item \textbf{WP1} \quad An educator upload correct test cases and automation scripts
    \item \textbf{WP1} \quad An educator evaluates the work done by the team at the end of the code kata battle
\end{description}
\par\noindent\rule{\textwidth}{0.5pt}


\subsection{Shared Phenomena}
\begin{description}
    \item \(SP_i\) An educator creates a tournament
    \item \(SP_i\) An educator grants other colleagues permission to create battles
    \item \(SP_i\) An educator creates a battle within a tournament
    \item \(SP_i\) The students are notified of a new tournament
    \item \(SP_i\) The students are notified of a new upcoming battle within a tournament they are subscribed to
    \item \(SP_i\) Students use the platform to form teams
    \item \(SP_i\) Student invite other students to join its team respecting the boundaries imposed
    \item \(SP_i\) Student join a battle on his own
    \item \(SP_i\) Student join a battle towards an invite -> Possono scegliere di essere soli oppure devono per forza
    \item \(SP_i\) The platform sends the link of the GitHub repository to all students who are members of subscribed teams
    \item \(SP_i\) Students are asked to fork the GitHub repository and set up an automated workflow
    \item \(SP_i\) The forked repository's workflow notifies the platform of a new push
    \item \(SP_i\) The platform updates the battle score of a team
    \item \(SP_i\) The educator uses the platform to go through the sources produced by each team
    \item \(SP_i\) The educator assigns additional score to the teams after the evaluation
    \item \(SP_i\) The platform notifies the teams when the final battle rank becomes available
    \item \(SP_i\) The platform updates the personal tournament score of each student
    \item \(SP_i\) An educator closes a tournament
    \item \(SP_i\) The platform notifies all students involved in the tournament about its end
    \item \(SP_i\) All users see the list of  ongoing tournaments as well as the corresponding tournament rank
    \item \(SP_i\) An educator defines the gamification badges
    \item \(SP_i\) Student visualize gained badges on its profile
\end{description}

\section{C. Definitions, Acronyms, Abbreviations}
\subsection{Abbreviations}
\(G_i\) = i-th goal \\
\(WP_i\) = i-th World Phenomena\\
\(SP_i\) = i-th Shared Phenomena\\

\subsection{Acronyms}
E = educator \\
S = students
\section{Revision history}
\section{Reference Documents}
\section{Document Structure}