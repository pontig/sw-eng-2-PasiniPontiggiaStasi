\chapter{Introduction}

\section{Purpose}
Software development is one of the most sought-after skills globally, and its growing importance is setted to persist in during future years. 
The versatility of this ability allows solving many problems in different fields, including science, art, and entrepreneurship.
The acquisition of software development skills improves both creativity and innovation, as well as fosters the development of fundamental skills such as problem-solving, logical thinking, and effective communication. 
To understand and master software development, it is essential to have the ability to design, write, and test software programs.
Dedication and patience are necessary for the gradual process of learning programming.
Theory is a foundational element, but it is only the start of developing your skills. Consistent practice is necessary to master concepts and programming techniques.

\subsection{Purpose of the document}
The aim of this document is to give a comprehensive explanation of the requirements and specifications for the project.
Functional and non-functional requirements are provided, with a focus on real possible use cases, end user interaction, and system limits and constraints. 
Those who will read this document include end users who are primarily interested in a high-level description of the features 
and developers who must implement the requirements presented in this document.

\subsection{Purpose of the product}
The purpose of the product is to provide a solution to the problem previously highlighted.
The \textbf{C}ode\textbf{K}ata\textbf{B}attle project, or CKB, is a platform that assist students in enhancing their software development abilities through battles.
Students in teams participate in programming exercises where they complete software projects following a test-first approach.
Creating battles, setting rules, and evaluating students' performance are tasks that educators do.
CKB promotes skill development, competition, and evaluation, fostering a collaborative and skill-building atmosphere.

\subsection{Goals}
The following table provides an aggregate list of the specific goals that must be accomplished by CodeKataBattle system.

\begin{table}[H]
    \centering
    \renewcommand{\arraystretch}{1.5} 
    \begin{tabular}{l l p{12.5cm}}
    \hline
        \textbf{G1} & & Educators can create tournaments that involve coding battles to challenge students. \\                                                                                                
        \textbf{G2} & & Provides educators with the ability to track student software development knowledge, evaluate the code written during battles, and monitor the overall tournament score. \\ 
        \textbf{G3} & & Educators can customize battles by defining specific rules, achievement objectives, and evaluation methodology. \\
        \textbf{G4} & & Students can improve software development skills by taking part in coding tournaments and battles where they must write programs. \\
        \textbf{G5} & & Coding battles enable students to enhance their soft skills, such as communication, collaboration, and time management, by creating team and collaborating with the members. \\
    \hline
    \end{tabular}
    \caption{Goals}
\end{table}

\section{Scope}

{\color{red} In another RASD here there are a few words}

\subsection{World Phenomena}

\begin{table}[H]
    \centering
    \renewcommand{\arraystretch}{1.5} 
    \begin{tabular}{l l p{12cm}}
    \hline
        \textbf{WP1} & & The students fork the GitHub repository of the code kata. \\                                                                                                
        \textbf{WP2} & & Students set up an automatic workflow in the GitHub repository. \\ 
        \textbf{WP3} & & The students work on the code kata battle. \\
        \textbf{WP4} & & The students push their work to the GitHub repository. \\
        \textbf{WP5} & & An educator upload correct test cases and automation scripts. \\
        \textbf{WP6} & & An educator evaluates the work done by the team at the end of the code kata battle. \\
    \hline
    \end{tabular}
    \caption{World Phenomena}
\end{table}

\subsection{Shared Phenomena}

\renewcommand{\arraystretch}{0.5} 
\begin{longtable}[H]{l l p{8.5cm} l l}
    \hline 
                      & & & & \\
        \textbf{ID}   & \vline & \textbf{Description} & \vline & \textbf{Controlled by} \\   
                      & & & & \\\hline & & & & \\
        \textbf{SP1}  & \vline & An educator creates a tournament & \vline & World \\
                      & & & & \\\hline & & & & \\
        \textbf{SP2}  & \vline & An educator grants other colleagues permission to create battles & \vline & Machine\\
                      & & & & \\\hline & & & & \\
        \textbf{SP3}  & \vline & An educator creates a battle within a tournament & \vline & \\
                      & & & & \\\hline & & & & \\
        \textbf{SP4}  & \vline & The students are notified of a new tournament & \vline & \\
                      & & & & \\\hline & & & & \\
        \textbf{SP5}  & \vline & The students are notified of a new upcoming battle within a tournament they are subscribed to & \vline & \\
                      & & & & \\\hline & & & & \\
        \textbf{SP6}  & \vline & Students use the platform to form teams & \vline & \\
                      & & & & \\\hline & & & & \\
        \textbf{SP7}  & \vline & Student invite other students to join its team respecting the boundaries imposed & \vline & \\
                      & & & & \\\hline & & & & \\
        \textbf{SP8}  & \vline & Student join a battle on his own & \vline & \\
                      & & & & \\\hline & & & & \\
        \textbf{SP9}  & \vline & Student join a battle towards an invite -> Possono scegliere di essere soli oppure devono per forza & \vline & \\
                      & & & & \\\hline & & & & \\
        \textbf{SP10} & \vline & The platform sends the link of the GitHub repository to all students who are members of subscribed teams & \vline & \\
                      & & & & \\\hline & & & & \\
        \textbf{SP11} & \vline & Students are asked to fork the GitHub repository and set up an automated workflow & \vline & \\
                      & & & & \\\hline & & & & \\
        \textbf{SP12} & \vline & The forked repository's workflow notifies the platform of a new push & \vline & \\
                      & & & & \\\hline & & & & \\
        \textbf{SP13} & \vline & The platform updates the battle score of a team & \vline & \\
                      & & & & \\\hline & & & & \\
        \textbf{SP14} & \vline & The educator uses the platform to go through the sources produced by each team & \vline & \\
                      & & & & \\\hline & & & & \\
        \textbf{SP15} & \vline & The educator assigns additional score to the teams after the evaluation & \vline & \\
                      & & & & \\\hline & & & & \\
        \textbf{SP16} & \vline & The platform notifies the teams when the final battle rank becomes available & \vline & \\
                      & & & & \\\hline & & & & \\
        \textbf{SP17} & \vline & The platform updates the personal tournament score of each student & \vline & \\
                      & & & & \\\hline & & & & \\
        \textbf{SP18} & \vline & An educator closes a tournament & \vline & \\
                      & & & & \\\hline & & & & \\
        \textbf{SP19} & \vline & The platform notifies all students involved in the tournament about its end & \vline & \\
                      & & & & \\\hline & & & & \\
        \textbf{SP20} & \vline & All users see the list of  ongoing tournaments as well as the corresponding tournament rank & \vline & \\
                      & & & & \\\hline & & & & \\
        \textbf{SP21} & \vline & An educator defines the gamification badges & \vline & \\
                      & & & & \\\hline & & & & \\
        \textbf{SP22} & \vline & Student visualize gained badges on its profile & \vline & \\
                      & & & & \\
    \hline & & & & \\
\caption{Shared Phenomena}
\end{longtable}
    
\section{Definitions, Acronyms, Abbreviations}

\subsection{Definitions}
\begin{table}[H]
    \centering
    \renewcommand{\arraystretch}{1.5} 
    \begin{tabular}{l l p{11.5cm}}
    \hline
        \textbf{Term}       & & \textbf{Definition} \\                                                                                                
    \hline
        Educator            & & Identifies a person who provides instruction or education, such as a teacher. \\                                                                                                
        Student             & & Identifies a person who is studying at a school or college. \\ 
    \hline
    \end{tabular}
    \caption{Definitions}
\end{table}

\subsection{Acronyms}
\begin{table}[H]
    \centering
    \renewcommand{\arraystretch}{1.5} 
    \begin{tabular}{l l p{11cm}}
    \hline
        \textbf{Acronyms}   & & \textbf{Term} \\                                                                                                
    \hline
        CKB                 & & CodeKataBattle \\                                                                                                
        EDU                 & & Educator \\ 
        STD                 & & Student \\ 
    \hline
    \end{tabular}
    \caption{Acronyms}
\end{table}

\subsection{Abbreviations}
\begin{table}[H]
    \centering
    \renewcommand{\arraystretch}{1.5} 
    \begin{tabular}{l l p{10.5cm}}
    \hline
        \textbf{Abbreviation}   & & \textbf{Term} \\                                                                                                
    \hline
        w.r.t                   & & With reference to \\                                                                                                
        \textbf{G}\(_i\)        & & i-th goal \\ 
        \textbf{WP}\(_i\)       & & i-th World Phenomena \\ 
        \textbf{SP}\(_i\)       & & i-th Shared Phenomena \\ 
        \textbf{R}\(_i\)        & & i-th Requirement \\ 
    \hline
    \end{tabular}
    \caption{Abbreviations}
\end{table}

\section{Reference Documents}
\begin{itemize}
    \item Assignment RDD A.Y 2023-2024
    \item Course slides on WeeBeep
    \item RASD review by Prof. M. Camilli
    \item ISO/IEC/IEEE 29148 dated 2018, Systems and software engineering - Life cycle processes - Requirements engineerings
\end{itemize}

\section{Document Structure}
The structure of this RASD document follows six main sections:
\begin{enumerate}
    \item \textbf{Introduction:}
          provides an overview of the problem at hand, the purpose of the document and
          the project, the scope of the domain, and introduces the main goals of the system as a solution.

    \item \textbf{Overall Description:}
          gives a general description of the system, going into more details about its main functions.
          The description is assisted with the help of UML diagrams, such as class, activity and state diagram.
          The domain assumptions of the examined world are then explained along with any dependencies and constraints.

    \item \textbf{Specific Requirements:}
          specifies the functional and non-functional requirements of a software system. 
          It includes use cases diagrams, descriptions of each use case, and related sequence diagrams. 
          Finally, it provides a mapping of the requirements to both goals and use cases.

    \item \textbf{Formal Analysis Using Alloy:}
          contains Alloy models which are used for the description of the application domain and his properties, 
          referring to the operations which the system has to provide and some critical aspects of the system.

    \item \textbf{Effort Spent:}
          keeps track of the time spent to complete this document.
          The first table defines the amount of hours used by the whole team to get important decisions and to make reviews,
          the other tables contains the individual effort spent by each team member.

    \item \textbf{References:}
          lists all the documents used and that were helpfull in drafting the RASD.
\end{enumerate}