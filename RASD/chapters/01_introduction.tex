\chapter{Introduction}

\section{Purpose}
The purpose of this document is to present a detailed description of CodeKataBattle (CKB).
It provides functional and non-functional requirements for the development of the system, including use cases, features, user interaction and system constraints. This document is addressed to the developers who have to implement the requirements and
could be used as an agreement between the customer and the contractors.

CodeKataBattle (CKB) is a new platform that helps students improve their software development skills by training with peers on code kata. Educators use the platform to challenge students by creating code
kata battles in which teams of students can compete against each other, thus proving (and improving) their skills and soft-skills : in fact working in a team is useful to learn soft-skills such as
communication and coordination.

A code kata battle is essentially a programming exercise in a programming language of choice (e.g.,
Java, Python). The exercise includes a brief textual description and a software project with build
automation scripts (e.g., a Gradle project in case of Java sources) that contains a set of test cases that
the program must pass, but without the program implementation. Students are asked to complete the
project with their code. In particular, groups of students participating in a battle are expected to follow
a test-first approach and develop a solution that passes the required tests. Groups deliver their
solution to the platform (by the end of the battle). At the end of the battle, the platform assigns scores
to groups to create a competition rank.

\subsection{Goals}
G Allows E to track S knowledge about sw development
G Allows E to create code kata tournaments and battles to challenge S
G Allows S to partecipate in code kata tournaments to improve their sw development skills
G Allows S to create teams for code kata battles to improves their soft skills
G Allows S to learn sw develoment divertendosi e confrontandosi con i compagni in sane competizioni con ranking
G At the end of the tournament, the platform shows the final ranks of each team

\blindtext
\section{B Scope}

\subsection{World Phenomena}
WP The students fork the GitHub repository of the code kata and set up an automatic workflow
WP The students work on the code kata battle
WP The students push their work to the GitHub repository
WP An educator upload correct test cases and automation scripts
WP An educator evaluates the work done by the team at the end of the code kata battle

\subsection{Shared Phenomena}
SP An educator creates a tournament
SP An educator grants other colleagues permission to create battles
SP An educator creates a battle within a tournament
SP The students are notified of a new tournament
SP The students are notified of a new upcoming battle within a tournament they are subscribed to
SP Students use the platform to form teams
SP Student invite other students to join its team respecting the boundaries imposed
SP Student join a battle on his own
SP Student join a battle towards an invite -> Possono scegliere di essere soli oppure devono per forza
SP The platform sends the link of the GitHub repository to all students who are members of subscribed teams
SP Students are asked to fork the GitHub repository and set up an automated workflow
SP The forked repository's workflow notifies the platform of a new push
SP The platform updates the battle score of a team
SP The educator uses the platform to go through the sources produced by each team
SP The educator assigns additional score to the teams after the evaluation
SP The platform notifies the teams when the final battle rank becomes available
SP The platform updates the personal tournament score of each student
SP An educator closes a tournament
SP The platform notifies all students involved in the tournament about its end
SP All users see the list of  ongoing tournaments as well as the corresponding tournament rank
SP An educator defines the gamification badges
SP Student visualize gained badges on its profile

\section{C. Definitions, Acronyms, Abbreviations}
\subsection{Abbreviations}
\(WP_i\) = i-th World Phenomena\\
\(SP_i\) = i-th Shared Phenomena\\
\(G_i\) = i-th goal
\subsection{Acronyms}
E = educator
S = students
\section{D. Revision history}
\section{E. Reference Documents}
\section{F. Document Structure}

CONTEGGIO ORE:
- Condivise:
i. W, G, SP, Scenari => 2h30m

- Elia
i.

- Michelangelo:
i.

- Tommaso:
i.
