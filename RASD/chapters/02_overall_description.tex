\chapter{Overall Description}
\section{Product Perspective}
\subsection{Scenarios}

\subsubsection{Creating a tournament}
Chip, a professor of Algorithm and Data Structures at Mouseton Institute of Technology, prepared to teach the chapter on strings, launching the "Strings Operations" coding tournament on CKB.
To expand participation, he allowed his colleague Dale to create challenges for his software engineering class.
Students across classes would compete in string manipulation tasks, ranging from basic concatenation to advanced text analysis, fostering collaboration and learning.
To make the tournament more interesting, Chip decided to award badges to the best performing students, so he created a badge for the student who participated in the most tournaments, one for the student who won the most battles and one for the student that wrote the most lines of code.
All students already subscribed to CKB were notified of the new tournament, and they could join it from the tournament page.

\subsubsection{Creating a battle}
In order to get the students familiar with the CKB platform and its features, Chip created an easy battle for his students to practice on, called "wordcheck", that basically requires the student to implement the game wordle.
He decides that the battle will last 2 weeks and that the students will be able to work in teams of 2 or 3 people; they will be able to join the battle until the last day of the battle.
In addiction, he wants to give extra points to the code cleanup, so he will have to revise the code of each team at the end of the battle and assign extra points to the teams that wrote clean code.
He sets all this information in the battle creation form and then he creates the battle.

\subsubsection{Joining a battle}
Huey and Dewey, two students of Chip's class, are notified of a new battle and decide to join it.
Since the the more the merrier, they decide to invite their friend Louie to join them in the battle.
After the registration deadline, they are notified that the battle is about to start and they are given the link to the GitHub repository of the battle to fork and set up the automated workflow in order to be able to link their GitHub account to the CKB platform.
After the automated workflow is set up, they are ready to start working on the battle.

\subsubsection{Improving the score and obtaining a badge}
Donald is a warrior of the "wordcheck" battle and he is working on the battle alone.
After a fitst commit, he logs in to the CKB to check his score.
He sees that he is in the 3rd position and that he is 10 points behind the leader group, that is composed by Huey, Dewey and Louie.
Fortunately, the battle is still open and the CKB platform allows him to improve his score by pushing new commits to the GitHub repository, so he decides to work on the battle for a couple of days and then push his work to the GitHub repository.
After checking his score again, he sees that he is now in the 1st position and moreover he obtained a badge for being the first to reach 100 points in the battle and now both students and professors can see this badge when they visit Donald's profile.

\subsubsection{Closing a battle}
When the deadline for the battle created by Scrooge is reached, all participants are notified that the battle is closed and that they can't push new commits to the GitHub repository.
Scrooge is notified that the battle is closed and he can now evaluate the code of each team and assign extra points for the clarity of the comments and the code, as he decided when he created the battle.
After the evaluation, the final rank of the battle is available to all participants and the students are notified that they can now see the final rank of the battle.

\subsubsection{Closing a tournament}
Chip decides to close the "Strings Operations" tournament, even if the deadline is not reached yet.
In order to do so, he logs in to the CKB platform and he closes the tournament.
All participants are notified that the tournament is closed and that they can't join it anymore, in the end, the CKB platform makes the final rank of the tournament available to all participants.

\subsubsection{Accessing the scores of the players}
Huey wishes to enroll to the class of Advanced Algorithms and Data Structures held by professor Pippo, so he applies for the class.
Pippo, who wants to make sure that Huey is a good student, comes to know that Huey is a very active user of the CKB platform and he decides to check his profile.
He sees that Huey has a very high score in the "Strings Operations" tournament and that he has a badge for being the most active user of the platform and he also notes that Huey is involved in more than one tournament simultaneously.
Thanks to the CKB platform, Pippo has now a complete overview of Huey's skills and he can decide whether to accept his application or not.