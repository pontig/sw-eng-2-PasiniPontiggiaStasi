\chapter{Specific Requirements}

\section{External Interface Requirements}
In the following, the interfaces, both hardware and software, of the system are described.

    {\color{red}\subsection{User Interfaces}}

\subsection{Hardware Interfaces}
To use the CKB platform, both the Educators and the Students need an electronic device connected to the Internet, like a computer, a tablet or a smartphone.\\
As the platform's primary functionality is closely tied to coding activities, it is expected that users will predominantly employ personal computers to access an Integrated Development Environment (IDE). Consequently, the platform's interfaces have been optimized for use on computer screens.\\

\subsection{Software Interfaces}
Since the platform is web-based, it is compatible with all the major operating systems, as long as they have a modern browser installed.

\subsection{Communication Interfaces}
The system requires a stable internet connection to work properly. The backend of the system will expose a unified RESTful API to communicate with all clients.\\
Furthermore, he system relies on various external interfaces accessible via uniform web API. These services are:
\begin{itemize}
    \item \textbf{GitHub API:} to create and manage repositories and to retrieve the code of the students and to authenticate users.
    \item \textbf{Mail API:} to send emails to the users to notify them about events.
\end{itemize}

{\color{red} \section{Functional Requirements}}
In order to work properly, the software must fulfill the following functional requirements:
    \begin{requirements}
        \item \textbf{Ri} \quad The software shall allow the students to create an account
        \item \textbf{Ri} \quad The software shall allow the educators to create an account 
        \item \textbf{Ri} \quad The software shall allow the students to login
        \item \textbf{Ri} \quad The software shall allow the educators to login
        \item \textbf{Ri} \quad The software shall allow the educators to create new tournaments
        \item \textbf{Ri} \quad The software shall allow an educator to grant to other colleagues the permission to create battles
        \item \textbf{Ri} \quad The software shall allow the educators allowed for a specific tournament to create coding kata battles in a that specific tournament by letting them uploading the code kata, setting minimum and maximum number of students per group, setting a registration and a final submission deadlines, setting score configurations
        \item \textbf{Ri} \quad The software shall allow the students to form teams {\color{red} Penso che dovrei essere più specifico e dire come creare i team ma d'altro canto non dovrei parlare come se conoscessi l'implementazione}
        \item \textbf{Ri} \quad The software shall send to all students who are members of subscribed teams to a kata battle the link to a GitHub repository containing the code kata {\color{red} Qui non so se dovrei parlare anche dell'interazione con GitHub per la creazioen delle repo}
        \item \textbf{Ri} \quad The software shall run the tests on executables pushed by a team and it shall also calculate and update the battle score of the corresponding team 
        \item \textbf{Ri} \quad The software shall allow the educator to manually evaluate the work done by students subscribed to his own kata battle and the software shall show the sources produced by each team
        \item \textbf{Ri} \quad At the end of the consolidation stage of a specific battle b, the software shall send a notification to all students partecipating to b when the final battle rank becomes available
        \item \textbf{Ri} \quad The software shall allow to all educators and students subscribed to CKB platform to see the personal tournament score of each student ( which is the sum of all battle scores received in that tournament) and a rank that measures how a student's performance compares to other students in the context of that tournament. They should also see the list of ongoing tournaments as well as the corresponding tournament rank.  
        \item \textbf{Ri} \quad The software shall allow an educator to close a tournament if and only if he is one of the owner of that tournament
        \item \textbf{Ri} \quad The software shall notify all students involved in a closed tournament whgen the final tournament rank becomes available
        \item \textbf{Ri} \quad When the educator creates a tournament, the software shall allow him to define gamification badges concerning that specific tournament 
        \item \textbf{Ri} \quad The software shall allow the educator to create new badges and define new rules as well as new variables associated with them in his own tournaments
        \item \textbf{Ri} \quad The software shall allow to all users to visualized the badges : in particular, both students and educators can see collected badges when they visualize the profile of a student
    
    \end{requirements}

\section{Performance Requirements}
To guarantee a good user experience, the system must:

\begin{itemize}
    \item Make sure the backend can grow as needed, respond quickly to changes, and balance the workload effectively.
    \item Be protected against DDoS attacks to keep the system safe and stable.
    \item Create a user-friendly, responsive front-end. It should handle well even when the internet isn't great, so users have a smooth experience.
    \item Send push notifications really quickly, so users don't even notice the delay.
\end{itemize}

{\color{red}\section{Design Constraints}}

\section{Software System Attributes}
\subsection{Reliability}
Since some functionality of the system relies on external APIs, though the system should not completely fail because of failure in one of those.\\
It's also important to avoid data loss through redundant storage methods.

\subsection{Availability}
In the event of an unplanned system downtime, all features should be restored as quickly as possible to minimize any inconvenience. To prevent such occurrences, it is crucial for the CKB platform to have a reliable infrastructure, including redundant servers, to ensure continuous operation.
The aimed availability of the system is 99.5\%, which means that the system can be down for a less than two days per year.\\
The system should also be able to handle a large number of concurrent users.

\subsection{Security}
Users of the system have distinct privileges according to their roles (student / educator), determined during the login process.\\
All data and information transferred and stored within the system are secured through robust encryption methods, such as HTTPS, ensuring data privacy and security.\\

\subsection{Maintainability}
The source code and associated documentation must include clear comments and should be consistently maintained. During the design and development phases, emphasis should be placed on achieving modularity, minimizing coupling, and ensuring high cohesion between components.
This is especially crucial for both the front-end and back-end, allowing developers to make updates to the back-end seamlessly without causing any disruptions or noticeable changes for users.\\
To avoid inconvenience in solving any type of problem (e.g. server downtime), maintenance services are notified to all users with an advance notice of at least 36 hours.

\subsection{Portability}
Due to the fact that the CKB platform is a distributed system, and it doesn't rely on a specific hardware or software, it can be used / accessed in multiples way.\\

{\color{red}\subsection*{Summary of Non-Functional Requirements}}
