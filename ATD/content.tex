\chapter{Introduction}

This document has the purpose of showing the results of the acceptance tests performed on the CodeKataBattle platform developed by another team. The tests are performed on the system to verify that the system meets the requirements and the expectations of the customer.

In the following pages it is reported a complete and detailed description of the tests performed, the results obtained and the bugs found during the tests.

\section{Scope}

The project analyzed was developed by the team composed by:
\begin{itemize}
    \item Maria Lucia Porfido
    \item Rosanna Iannaccone
    \item Federica Persico
\end{itemize}
Team's repository: \url{https://github.com/marialucia54/IannacconePersicoPorfido}

\chapter{Installation and setup}

Although the installation instruction reported on the ITD document are complete and exhaustive, the code presents a README file that briefly explains how to install and run the system (if the machine is already equipped with python), and the major dependencies are automatically installed by simply running the \texttt{pip install -r requirements.txt} command.\\

All the installation went smoothly, and the system was up and running in a few minutes.\\The only thing that should be added is a default location for the application, because if the user directly connects to \texttt{localhost:8000} as suggested by the server log, no page is shown, and the user has to manually navigate to \texttt{localhost:8000/login} to access the login page.


\chapter{Acceptance test cases}

The test cases are taken entirely from the use cases reported in the RASD document. The test cases are reported in the following tables: \\{\color{red} GITHUB}

\section{Test cases}
\begin{center}
    \def\arraystretch{1.5}
    \begin{tabular}{|m{2cm} |m {10cm}|}
        \hline
        Name      & Registration on CKB platform                                           \\ \hline
        Input     & The user inserts its credentials and clicks on the registration button \\ \hline
        Expected  & The user is registered on the platform and can access the main page    \\ \hline
        Pass/Fail & Pass                                                                   \\ \hline
    \end{tabular}
\end{center}

\begin{center}
    \def\arraystretch{1.5}
    \begin{tabular}{|m{2cm} |m {10cm}|}
        \hline
        Name      & Discovering and Participating in Tournaments                                                                                                                                                                                                                                                                                                            \\ \hline
        Input     & The STU navigates on the plaform and selects and joins a tournament                                                                                                                                                                                                                                                                                     \\ \hline
        Expected  & The STU is now part of the tournament                                                                                                                                                                                                                                                                                                                   \\ \hline
        Pass/Fail & Incomplete                                                                                                                                                                                                                                                                                                                                              \\ \hline
        Notes     & Once the student selects the tournament on the main page, the system will automatically add the student to the tournament; it would be better to add a confirmation message to the user (as it is expected to happen according to what is written on the RASD use case). Furthermore, nowhere it is specified the submission deadline to the tournament \\ \hline
    \end{tabular}
\end{center}

\begin{center}
    \def\arraystretch{1.5}
    \begin{tabular}{|m{2cm} |m {10cm}|}
        \hline
        Name        & Creating a Tournament                                                                           \\ \hline
        Input       & The EDU creates a tournament and fills in the required fields, then clicks on the create button \\ \hline
        Expected    & The tournament is created                                                                       \\ \hline
        Pass / Fail & Pass                                                                                            \\ \hline
    \end{tabular}
\end{center}

\begin{center}
    \def\arraystretch{1.5}
    \begin{tabular}{|m{2cm} |m {10cm}|}
        \hline
        Name        & Granting permissions                                                                                                                                                    \\ \hline
        Input       & In a tournament, created by edu A, A can search for another EDU B and grant him the permission to manage the tournament                                                 \\ \hline
        Expected    & EDU B is now a manager of the tournament                                                                                                                                \\ \hline
        Pass / Fail & Fail: it is not clear what to insert in the text box (email, just name, name and surname, github username...), and by trying to put any ot these it doesn't work anyway \\ \hline
    \end{tabular}
\end{center}

\begin{center}
    \def\arraystretch{1.5}
    \begin{tabular}{|m{2cm} |m {10cm}|}
        \hline
        Name        & Joining a battle                                          \\ \hline
        Input       & The STU selects a tournament and joins a battle within It \\ \hline
        Expected    & The STU is now part of the battle                         \\ \hline
        Pass / Fail & Pass                                                      \\ \hline
    \end{tabular}
\end{center}

\begin{center}
    \def\arraystretch{1.5}
    \begin{tabular}{|m{2cm} |m {10cm}|}
        \hline
        Name        & Viewing tournament rankings                                   \\ \hline
        Input       & The User selects a tournament and selects the rankings button \\ \hline
        Expected    & The User can see the rankings of the tournament               \\ \hline
        Pass / Fail & Pass                                                          \\ \hline
    \end{tabular}
\end{center}

\begin{center}
    \def\arraystretch{1.5}
    \begin{tabular}{|m{2cm} |m {10cm}|}
        \hline
        Name        & Closing a tournament                       \\ \hline
        Input       & The EDU selects a tournament and closes it \\ \hline
        Expected    & The tournament is closed                   \\ \hline
        Pass / Fail & Pass                                       \\ \hline
    \end{tabular}
\end{center}




\section{Bug reports}

There are quite a few recurring bugs in the system, which are listed below:
\begin{itemize}
    \item The system does not show the message to confirm that a general action (e.g. subscription to a tournament, creation / closure of a tournament, etc.) has been completed: this is as simple as important to the user experience, as it is expected to happen according to what is written on the RASD use case.
    \item Whenever an error occurs, the system does not show any message to the user, but it simply dumps the python error page on the screen: this is not only bad for the user experience, but it is also a security issue, as it could reveal sensitive information to the user \\ The stack dump is very useful for the developers, but it should be removed after the development phase.
    \item The students aren't always notified relevant events occur (the \textit{Notification} section doesn't show the new entries)
    \item {\color{red} Qualcuno riesce a connetersi a fucking github?!}
\end{itemize}

\section{Final Comments}

In general, we found the application usable and quite user friendly, with a well-structured code and sufficiently commented in large files \\

We think that the only problems we found out are due to the short time the project was developed and we are quite sure that, in a real world scenario, after out revision the developers would immediatly fix the bugs we found out; we are not so familiar with python but we still think that it is just a matter of a bunch of lines of code