\chapter{Description of the software}

This document contains the description of the implementation of the CodeKataBattle platform.

Its purpose is to provide a complete and detailed explanation of the system and its structure to the developers who will work on the project (in our case, the members of other teams and our professors). It is also a useful reference for the developers who will maintain the system in the future.

\section{Links}
In the following list are provided the links to all the documents that are part of the CodeKataBattle project's implementation

\begin{itemize}
    \item \href{https://github.com/pontig/sw-eng-2-PasiniPontiggiaStasi/tree/main/ITD/CodeKataBattle}{Source Code}
    \item \href{https://github.com/pontig/sw-eng-2-PasiniPontiggiaStasi/tree/main/ITD/CodeKataBattle/out/artifacts/CodeKataBattle_jar}{executable ready to be run}
\end{itemize}

\section{Implemented features}

\input implemented

\section{Adopted development framework}

For a thorough exposition of the pros and cons associated with the languages in question, it is recommendend to search directly on the official websites of said languages, where such aspects are explicated in detail

\subsubsection{Backend}
The system has been developed using the \textbf{Spring Boot} framework, which is a framework that allows to create web applications in Java.\\
The main reason for this choice is that it is a framework often used in this context since it allows to create a web application with REST APIs in a simple way.\\In addiction, there are a lot of libraries that can be used to integrate the framework with other technologies, such as the database, which is very useful.\\

In order to perform the static analysis of the code, we have used \textbf{SonarQube}, which is a tool that allows to perform a static analysis of the code and to find the bugs and the vulnerabilities.

\subsubsection{Frontend}
For the development of the frontend, we have implemented a single page application simply using \textbf{HTML, SCSS and JavaScript}.\\Even it is a dated approach, we have chosen this solution because it is the one that we are most familiar with and because it is the one that we have used in the previous projects.\\Moreover, since the application is very simple, we have not felt the need to use a more complex framework.

\subsubsection{Database}
The database has been implemented using \textbf{MySQL}.\\
The main reason for this choice is that it is the most used database in the world and it can be easily integrated with the Spring Boot framework thanks to the \textbf{Spring Data JPA} library.

\section{Structure of the source code}

The source code of the system can be found in the \href{https://github.com/pontig/sw-eng-2-PasiniPontiggiaStasi/tree/main/ITD/CodeKataBattle}{ITD folder} of the repository.\\
Its structure is the classical one of a Spring Boot application, which is the following:
\begin{itemize}
    \item {\ttfamily{src/main/java/ckb/platform/}}: contains the source code of the spring application, divided in the following subfolders:
          \begin{itemize}
              \item {\ttfamily{advices/}}: contains the classes that are used to insert into the response the error messages in case of exceptions
              \item {\ttfamily{controllers/}}: contains the classes that are used to handle the requests and are the responsible of the communication between the APIs and the database
              \item {\ttfamily{entities/}}: contains the JPA classes that represent the entities of the database
              \item {\ttfamily{exceptions/}}: contains the classes that are used to handle the exceptions
              \item {\ttfamily{formParser/}}: contains the java beans that are used to parse the forms of the POST requests
              \item {\ttfamily{githubAPI/}}: contains the class that is used to communicate with the GitHub API
              \item {\ttfamily{repositories/}}: contains the interfaces that are used to communicate with the database
              \item {\ttfamily{scheduler/}}: contains the classes that handle the additional threads that keeps the time updated and, whenever a deadline is reached, that send the emails to the interested users
              \item {\ttfamily{testRepo/}} contains the class that is the responsible of the building of the code and the execution of the tests
          \end{itemize}
    \item {\ttfamily{src/main/resources/static/}}: contains the files of the frontend application, i.e. the HTML, SCSS and JavaScript files
\end{itemize}

\chapter{Performed test}

The software testing procedure has been performed mainly thorough unit testing; below are listed the most relevant tests that have been performed.\\
Due to the limited time, we have focused on the backend system, since the frontend is very simple and the majority of the functionalities are implemented in the backend.\\

\textbf{Note}: Regarding any kind of misuse of the system, the frontend already has some all the necessary checks to avoid any kind of misuse of the system (for instance, it is not possible to subscribe to a tournament that is already closed, it is not possible to subscribe to a battle that is already closed, but also all form is equipped with the necessary type checks and length checks).\\
Furthermore, in the RASD we have already assumed from the domain that the user won't try to misuse the system.\\
Nonetheless, the backend provides all the checks already implemented in the frontend, so that the system is safe from any kind of misuse, and returns the right error messages in case of misuse.

\section{Tournament}

\begin{enumerate}
    \item an EDU creates a tournament, with a new name or a name that already exists
    \item a STU subscribes to a tournament
    \item an EDU closes a tournament
\end{enumerate}

\section{GitHub / Static Analysis / mail sending}

\begin{enumerate}
    \item the system sends an email to the right users when a deadline is reached or a significant event happens
    \item the system creates a GitHub repository at the end of the registration phase of a battle
    \item after a push, the system gets notified by GitHub, retrieves the code in the form of a zip file, unzips it, runs the tests and updates the score
\end{enumerate}

\section{Battle}

\begin{enumerate}
    \item an EDU creates a battle, selecting one or more options regarding static analysis and manual evaluation
    \item a STU subscribes to a battle, both alone and inviting other STUs
    \item a STU invites another STU to a battle after being subscribed
    \item a STU joins a battle by an invite
    \item a STU uploads the code on GitHub and sees his score updated, taking into account the tests, the timeliness and the eventual static analysis
    \item an EDU performs the manual evaluation of the code of the teams in a battle
\end{enumerate}

\section{Profile}

\begin{enumerate}
    \item an user inspects the profile of a STU's profile showing the list of tournaments and battles the student is subscribed to
\end{enumerate}

\chapter{Installation instructions}

\section{Java}
In order to run the application, it is necessary to have Java installed on the machine.
\begin{itemize}
    \item \textbf{Java Runtime Environment}: \href{https://www.java.com/it/download/manual.jsp}{Download JDK}: at least version 21
    \item \textbf{Java Development Kit}: \href{https://www.oracle.com/java/technologies/downloads/}{Download JRE}: at least version 17
\end{itemize}

Once the installation is completed, it is necessary to set the environment variables (the way to do this depends on the operating system).\\

\section{MySQL}
In order to run the application, it is necessary to have MySQL installed on the machine and to create a database.\\
You can download it from the \href{https://dev.mysql.com/downloads/installer/}{official website}.\\
Required modules to be installed:
\begin{itemize}
    \item MySQL Server
    \item MySQL Workbench (optional, but recommended)
    \item Connector/J
\end{itemize}
Once the installation is completed, it is necessary to create a schema called \textbf{`ckb-db`}, and create a user with all the privileges on this schema and with the following credentials:
\begin{itemize}
    \item username: \textbf{CKBPlatform}
    \item password: \textbf{CKB202430L!}
\end{itemize}

In order to work as expected, the database must be running on the default port (3306).

\section{Static Analysis Tool - SonarQube}
In order to use this tool, we need to install SonarQube Community Edition at this \href[]{https://www.sonarqube.org/downloads/}{link}.\\
Then, install a local instance of SonarQube by following these \href[]{https://docs.sonarsource.com/sonarqube/latest/try-out-sonarqube/#installing-a-local-instance-of-sonarqube}{instructions}.\\
In our case we simply run from the zip file, using Windows as a OS and we have installed it in the same directory of the instructions.\\

Notice that if you are installing it in a different location, you have to insert the correct directory when the application starts.\\
\textbf{Java 17 is required to start the local instance of SonarQube}.\\
Last thing, install the SonarScanner for CLI and we can find the download \href[]{https://docs.sonarsource.com/sonarqube/latest/analyzing-source-code/scanners/sonarscanner/}{here}.

\section{GitHub API}

As our system runs on  \url{https://localhost:8080} and GitHub need to reach a public address, we will use "ngrok". Ngrok generates a tunnel address that need to be used in the WorkFlow Action to reach the End-Point \texttt{/ckb{\_}platform/battle/pulls}

In order to make ngrok work properly, follow these instructions:
\begin{itemize}
    \item Log In with CKB GitHub credentials (mail: \textbf{codekatabattle.platform@gmail.com} passeord: \textbf{CKB202430l!}) on Ngrok \url{https://dashboard.ngrok.com/login}
    \item Go in section Your Authtoken and copy the command: \texttt{ngrok config add-authtoken 2bTBjPXyFBY3dy8NlZSJ5LzrC9x{\_}7FRwMrsggt8f68KFpvJfW}
    \item Ngrok can be downloaded from \url{https://ngrok.com/download}, unzip it, and open it (CMD will pop up)
    \item Paste the command you copied before
\end{itemize}
From now, every time you need to use ngrok, open it and type the command: \texttt{ngrok http https://localhost:8080}

A public address will be prompted, this will be used in GitHub Action\\

In conclusion, whenever a GitHub repo is created and forked, in the readme you will find a template for the action workflow, you simply have to put the tunnel link where you find \texttt{<<NGROK>>}

\section{pytest}

In order to comput the test score in a python battle, the platform uses \texttt{pytest}.\\

The installation is quite straight-forward if you have already python on your machine: it is sufficient to run the command \texttt{pip install --upgrade --force-reinstall pytest}\\
If a warning appears showing a path quute long, it means that that path isn't among the environment variables: it is recommended to add it.

\section{Running the application}
Once all the previous steps are completed, it is possible to run the application.\\
To run the application, just run the jar file that can be found in the link above and follow the instructions given by the application.

\subsection{Some notes about the Gmail API}

To send mails using gmail to other addresses, Google requires an OAuth authentication system since 2019.\\
The application from Google is seen as if it were in the testing phase, so we have a token that expires every 7/10 days and must be renewed. We have not made it public because Google requires a series of checks regarding privacy/security etc.\\

The first time the application is run, it will open a link to which you must access with the credentials provided in the email.\\
After that, the token will be generated and the application will be able to send emails.\\
After 7/10 days, the token will expire and the application will show again the link to access to generate a new token.\\