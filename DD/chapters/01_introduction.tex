\chapter{Introduction}

\section{Purpose}
This document contains the complete design description of the CodeKataBattle platform. 
It provides a technical overview of the Requirement Analysis and Specification Document (RASD) for the system-to-be, describing the main architectural components, communication interfaces, and interactions. 
It also presents the implementation, integration, and testing plan. 
This document is mainly addressed to developers, testers, and project manager since it guides during the development process through an accurate vision of all parts of the software-to-be. 

\section{Scope}

\section{Definitions, Acronyms, Abbreviations}
\subsection{Definitions}
\begin{table}[H]
    \centering
    \renewcommand{\arraystretch}{0.5}
    \begin{tabular}{l l p{10cm}}
        \hline
                            &        &                                                                                                                        \\
        \textbf{Term}       & \vline & \textbf{Definition}                                                                                                    \\
                            &        &                                                                                                                        \\\hline \\
        User / Actor        & \vline & A person who uses CKB platform, could be a Student or an Educator.                                                     \\       
                            &        &                                                                                                                        \\\hline \\
        Educator            & \vline & Identifies a person who provides instruction or education, such as a teacher.                                          \\
                            &        &                                                                                                                        \\\hline \\
        Student             & \vline & Identifies a person who is studying at school or college.                                                              \\
                            &        &                                                                                                                        \\\hline \\
        Kata                & \vline & A training exercise system for karate where you repeat a form multiple times, making small improvements to each one.   \\
                            &        &                                                                                                                        \\\hline \\
        Test-first approach & \vline & A software development process based on converting software requirements into test cases before creating the software,
                                       and then tracking the entire development process by repeatedly testing the software against those test cases.          \\
                            &        &                                                                                                                        \\\hline \\
        Code Kata           & \vline & Programming battles in which teams of students compete against each other.                                             \\
                            &        &                                                                                                                        \\\hline \\
        Tier                & \vline & Physical components or servers that constitute a system.                                                               \\
                            &        &                                                                                                                        \\\hline \\
        Layer               & \vline & Logical separation of elements with associated functionality, such as presentation, business logic, and data access, 
                                       each carrying out specific tasks within an application.                                                                \\
                            &        &                                                                                                                        \\
        \hline
    \end{tabular}
    \caption{Definitions}
\end{table}

\subsection{Acronyms}
\begin{table}[H]
    \centering
    \renewcommand{\arraystretch}{0.5}
    \begin{tabular}{l l p{11cm}}
        \hline
                          &        &                                                        \\
        \textbf{Acronyms} & \vline & \textbf{Term}                                          \\
                          &        &                                                        \\\hline \\
        CKB               & \vline & CodeKataBattle                                         \\
                          &        &                                                        \\\hline \\
        EDU               & \vline & Educator                                               \\
                          &        &                                                        \\\hline \\
        STU               & \vline & Student                                                \\
                          &        &                                                        \\\hline \\
        IDE               & \vline & Integrated Development Environment                     \\
                          &        &                                                        \\\hline \\
        ESP               & \vline & Email Service Provider                                 \\
                          &        &                                                        \\\hline \\
        RASD              & \vline & Requirement Analysis and Specification Document        \\
                          &        &                                                        \\\hline \\
        UML               & \vline & Unified Modeling Language                              \\
                          &        &                                                        \\\hline \\
        API               & \vline & Application Programming Interface                      \\
                          &        &                                                        \\\hline \\
        UI                & \vline & User Interface                                         \\
                          &        &                                                        \\\hline \\
        UX                & \vline & User Experience                                        \\
                          &        &                                                        \\\hline \\
        OS                & \vline & Operating System                                       \\
                          &        &                                                        \\\hline \\
        REST              & \vline & Representational State Transfer                        \\
                          &        &                                                        \\\hline \\
        HTTPS             & \vline & HyperText Transfer Protocol Secure                     \\
                          &        &                                                        \\\hline \\
        JSON              & \vline & JavaScript Object Notation                             \\
                          &        &                                                        \\\hline \\
        DB                & \vline & DataBase                                               \\
                          &        &                                                        \\\hline \\
        DBMS              & \vline & DataBase Management Systems                            \\
                          &        &                                                        \\
        \hline
    \end{tabular}
    \caption{Acronyms}
\end{table}

\subsection{Abbreviations}
\begin{table}[H]
    \centering
    \renewcommand{\arraystretch}{0.5}
    \begin{tabular}{l l p{10.5cm}}
        \hline
                              &        &                        \\
        \textbf{Abbreviation} & \vline & \textbf{Term}          \\
                              &        &                        \\\hline & & \\
        \textbf{R}\(_i\)      & \vline & i-th Requirement       \\
                              &        &                        \\\hline & & \\
        \textbf{i.e.}         & \vline & in other words         \\
                              &        &                        \\\hline & & \\
        \textbf{e.g.}         & \vline & for example            \\
                              &        &                        \\\hline & & \\
        \textbf{iff}          & \vline & if and only if         \\
                              &        &                        \\
        \hline
    \end{tabular}
    \caption{Abbreviations}
\end{table}

\section{Reference Documents}
\begin{itemize}
    \item CodeKataBattle RASD\footnote{\url{https://github.com/pontig/sw-eng-2-PasiniPontiggiaStasi/blob/main/RASD/RASD.pdf}}
    \item Assignment RDD A.Y. 2023-2024\footnote{\url{https://webeep.polimi.it/mod/folder/view.php?id=219353}}
    \item Course slides on WeeBeep\footnote{\url{https://webeep.polimi.it/mod/folder/view.php?id=207692}}
    \item ISO/IEC/IEEE 29148 dated 2018, \\
          Systems and software engineering - Life cycle processes - Requirements engineering\footnote{\url{https://www.iso.org/obp/ui/\#iso:std:iso-iec-ieee:29148:ed-2:v1:en}}
\end{itemize}

\section{Document Structure}
The structure of this DD document follows six main sections:
\begin{enumerate}
    \item \textbf{Introduction:}
          provides an overview of the problem at hand, the purpose of the project,
          the scope of the domain, and introduces the main goals of the system as a solution.

    \item \textbf{Architectural Design:}
          gives a general description of the system, going into more detail about its main functions.
          The description is assisted with the help of UML diagrams, such as class, activity, and state diagrams.
          The domain assumptions of the examined world are then explained along with any dependencies and constraints.

    \item \textbf{User Interface Design:}
          specifies the functional and non-functional requirements of a software system.
          It includes use case diagrams, descriptions of each use case, and related sequence diagrams.
          Finally, it provides a mapping of the requirements for both goals and use cases.

    \item \textbf{Requirements Traceability:}
          provides a cross-reference that connects system components to the requirements specified in the Requirements Analysis and Specification Document (RASD). 
          This cross-reference is presented in a tabular format, which clearly shows which system components fulfill each of the functional requirements defined in the RASD.

    \item \textbf{Implementation, Integration, and Test Plan:}   
          details the strategy and sequence for deploying subsystems and components. 
          It identifies the order in which sub-components are implemented, integrated, and tested. Additionally, it provides a comprehensive explanation of the methodologies and technologies to employ throughout the process.

    \item \textbf{Effort Spent:}
          keeps track of the time spent to complete this document.
          The first table defines the amount of hours used by the whole team to make important decisions and to make reviews,
          the other tables contain the individual effort spent by each team member.
\end{enumerate}