\chapter{Introduction}

\section{Purpose}
This document contains the complete design description of the CodeKataBattle platform. 
It provides a technical overview of the Requirement Analysis and Specification Document (RASD) for the system-to-be, describing the main architectural components, communication interfaces, and interactions. 
It also presents the implementation, integration, and testing plan. 
This document is mainly addressed to developers, testers, and project manager since it guides during the development process through an accurate vision of all parts of the software-to-be. 

\section{Scope}


\section{Definitions, Acronyms, Abbreviations}
\subsection{Definitions}
\begin{table}[H]
    \centering
    \renewcommand{\arraystretch}{0.5}
    \begin{tabular}{l l p{10cm}}
        \hline
                            &        &                                                                                                                        \\
        \textbf{Term}       & \vline & \textbf{Definition}                                                                                                    \\
                            &        &                                                                                                                        \\\hline \\
        User / Actor        & \vline & A person who uses CKB platform, could be a Student or an Educator.                                                     \\                                                                                                                                                                                                                                 \\
                            &        &                                                                                                                        \\\hline \\
        Educator            & \vline & Identifies a person who provides instruction or education, such as a teacher.                                          \\
                            &        &                                                                                                                        \\\hline \\
        Student             & \vline & Identifies a person who is studying at school or college.                                                              \\
                            &        &                                                                                                                        \\\hline \\
        Kata                & \vline & A training exercise system for karate where you repeat a form multiple times, making small improvements to each one.   \\
                            &        &                                                                                                                        \\\hline \\
        Test-first approach & \vline & A software development process based on converting software requirements into test cases before creating the software,
                                       and then tracking the entire development process by repeatedly testing the software against those test cases.          \\
                            &        &                                                                                                                        \\\hline \\
        Code Kata           & \vline & Programming battles in which teams of students compete against each other.                                             \\
                            &        &                                                                                                                        \\\hline \\
        Tier                & \vline & Programming battles in which teams of students compete against each other.                                             \\
                            &        &                                                                                                                        \\
        
        
        \hline
    \end{tabular}
    \caption{Definitions}
\end{table}


\subsection{Acronyms}
\begin{table}[H]
    \centering
    \renewcommand{\arraystretch}{0.5}
    \begin{tabular}{l l p{11cm}}
        \hline
                          &        &                                    \\
        \textbf{Acronyms} & \vline & \textbf{Term}                      \\
                          &        &                                    \\\hline & & \\
        CKB               & \vline & CodeKataBattle                     \\
                          &        &                                    \\\hline & & \\
        EDU               & \vline & Educator                           \\
                          &        &                                    \\\hline & & \\
        STU               & \vline & Student                            \\
                          &        &                                    \\\hline & & \\
        IDE               & \vline & Integrated Development Environment \\
                          &        &                                    \\\hline & & \\
        ESP               & \vline & Email Service Provider             \\
                          &        &                                    \\
        RASD
        API 
                          
        \hline
    \end{tabular}
    \caption{Acronyms}
\end{table}


\subsection{Abbreviations}
\begin{table}[H]
    \centering
    \renewcommand{\arraystretch}{0.5}
    \begin{tabular}{l l p{10.5cm}}
        \hline
                              &        &                        \\
        \textbf{Abbreviation} & \vline & \textbf{Term}          \\
                              &        &                        \\\hline & & \\
        \textbf{R}\(_i\)      & \vline & i-th Requirement       \\
                              &        &                        \\\hline & & \\
        \textbf{i.e.}         & \vline & in other words         \\
                              &        &                        \\\hline & & \\
        \textbf{e.g.}         & \vline & for example            \\
                              &        &                        \\\hline & & \\
        \textbf{iff}          & \vline & if and only if         \\
                              &        &                        \\
        \hline
    \end{tabular}
    \caption{Abbreviations}
\end{table}

\section{Reference Documents}
\begin{itemize}
    \item Assignment RDD A.Y. 2023-2024\footnote{\url{https://webeep.polimi.it/mod/folder/view.php?id=219353}}
    \item Course slides on WeeBeep\footnote{\url{https://webeep.polimi.it/mod/folder/view.php?id=207692}}
    \item ISO/IEC/IEEE 29148 dated 2018, \\
          Systems and software engineering - Life cycle processes - Requirements engineering\footnote{\url{https://www.iso.org/obp/ui/\#iso:std:iso-iec-ieee:29148:ed-2:v1:en}}
\end{itemize}

\section{Document Structure}

{\color{red} Questa parte qui è mooooolto simile a quella che c'è nel RASD, quindi aspettiamo di avere il RASD definitivo per fare il merge}