\chapter{Introduction}

\section{Purpose}
This document contains the complete design description of the CodeKataBattle platform. 
It provides a technical overview of the Requirement Analysis and Specification Document (RASD) for the system-to-be, describing the main architectural components, communication interfaces, and interactions. 
It also presents the implementation, integration, and testing plan. 
This document is mainly addressed to developers, testers, and project manager since it guides during the development process through an accurate vision of all parts of the software-to-be. 

\section{Scope}
As explained in the RASD, CKB is a versatile platform designed to enhance students' software development skills through coding challenges. 
Educators can create coding battles in tournaments, where students solve programming exercises using a test-first approach. 
The system allows educators to upload Code Kata, set parameters like group size limits and deadlines, and define scoring criteria.

Teams, adhering to size constraints, participate in battles and can switch between different tournaments. 
After the registration deadline, teams access a GitHub repository for their Code Kata, working through an automated workflow.

Scoring ranges from 0 to 100, considering automated evaluations (test case pass rate, timeliness, source code quality) and optional manual evaluations by educators. 
Scores update in real-time as students commit code. 
Educators can perform manual evaluations before sharing the final rank.

Each student receives a personal score, the sum of their battle scores within a tournament. 
Educators can delegate battle creation and introduce gamification badges for student recognition. 
The platform provides a comprehensive system for coding skill improvement and competition.

The proposed system will take the form of a Web Application, accessible through any modern web browser. 
It will feature two distinct views, one for EDU and another for STU, each designed to implement the specified features effectively.
For a comprehensive understanding of the functionalities available to end-users, please consult the RASD.
The system architecture is structured into three physically separated layers, strategically installed on different tiers to optimize performance and functionality. \\
These layers include:
\begin{enumerate}
    \item \textbf{Presentation Layer:} responsible for managing the presentation logic and facilitating all interactions with end-users.
    \item \textbf{Business Logic Layer:} tasked with overseeing the application functions provided by the system under development.
    \item \textbf{Data Layer:} responsible for managing secure data storage and controlling access to the stored information.
\end{enumerate} 

\section{Definitions, Acronyms, Abbreviations}
\subsection{Definitions}
\begin{table}[H]
    \centering
    \renewcommand{\arraystretch}{0.5}
    \begin{tabular}{l l p{10cm}}
        \hline
                            &        &                                                                                                                        \\
        \textbf{Term}       & \vline & \textbf{Definition}                                                                                                    \\
                            &        &                                                                                                                        \\\hline \\
        User / Actor        & \vline & A person who uses the CKB platform, could be a Student or an Educator.                                                 \\       
                            &        &                                                                                                                        \\\hline \\
        Educator            & \vline & Identifies a person who provides instruction or education, such as a teacher.                                          \\
                            &        &                                                                                                                        \\\hline \\
        Student             & \vline & Identifies a person who is studying at school or college.                                                              \\
                            &        &                                                                                                                        \\\hline \\
        Kata                & \vline & A training exercise system for karate where you repeat a form multiple times, making small improvements to each one.   \\
                            &        &                                                                                                                        \\\hline \\
        Test-first approach & \vline & A software development process based on converting software requirements into test cases before creating the software,
                                       and then tracking the entire development process by repeatedly testing the software against those test cases.          \\
                            &        &                                                                                                                        \\\hline \\
        Code Kata           & \vline & Programming battles in which teams of students compete against each other.                                             \\
                            &        &                                                                                                                        \\\hline \\
        Tier                & \vline & Physical components or servers that constitute a system.                                                               \\
                            &        &                                                                                                                        \\\hline \\
        Layer               & \vline & Logical separation of elements with associated functionality, such as presentation, business logic, and data access, 
                                       each carrying out specific tasks within an application.                                                                \\
                            &        &                                                                                                                        \\\hline \\
        Frontend            & \vline & Manages the visual appearance and interaction of the application with users, also called Presentation Layer.           \\
                            &        &                                                                                                                        \\\hline \\
        Backend             & \vline & Manages the logic and server operations, comprising the "Business Logic Layer" and the "Data Layer."                   \\
                            &        &                                                                                                                        \\
        \hline
    \end{tabular}
    \caption{Definitions}
\end{table}

\subsection{Acronyms}
\begin{table}[H]
    \centering
    \renewcommand{\arraystretch}{0.5}
    \begin{tabular}{l l p{11cm}}
        \hline
                          &        &                                                        \\
        \textbf{Acronyms} & \vline & \textbf{Term}                                          \\
                          &        &                                                        \\\hline \\
        CKB               & \vline & CodeKataBattle                                         \\
                          &        &                                                        \\\hline \\
        EDU               & \vline & Educator                                               \\
                          &        &                                                        \\\hline \\
        STU               & \vline & Student                                                \\
                          &        &                                                        \\\hline \\
        IDE               & \vline & Integrated Development Environment                     \\
                          &        &                                                        \\\hline \\
        ESP               & \vline & Email Service Provider                                 \\
                          &        &                                                        \\\hline \\
        RASD              & \vline & Requirement Analysis and Specification Document        \\
                          &        &                                                        \\\hline \\
        UML               & \vline & Unified Modeling Language                              \\
                          &        &                                                        \\\hline \\
        ER                & \vline & Entity relationship                                    \\
                          &        &                                                        \\\hline \\
        API               & \vline & Application Programming Interface                      \\
                          &        &                                                        \\\hline \\
        UI                & \vline & User Interface                                         \\
                          &        &                                                        \\\hline \\
        UX                & \vline & User Experience                                        \\
                          &        &                                                        \\\hline \\
        OS                & \vline & Operating System                                       \\
                          &        &                                                        \\\hline \\
        REST              & \vline & Representational State Transfer                        \\
                          &        &                                                        \\\hline \\
        SPA               & \vline & Single Page Application                                \\  
                          &        &                                                        \\\hline \\
        HTTPS             & \vline & HyperText Transfer Protocol Secure                     \\
                          &        &                                                        \\\hline \\
        JSON              & \vline & JavaScript Object Notation                             \\
                          &        &                                                        \\\hline \\
        DB                & \vline & DataBase                                               \\
                          &        &                                                        \\\hline \\
        DBMS              & \vline & DataBase Management Systems                            \\
                          &        &                                                        \\
        \hline
    \end{tabular}
    \caption{Acronyms}
\end{table}

\subsection{Abbreviations}
\begin{table}[H]
    \centering
    \renewcommand{\arraystretch}{0.5}
    \begin{tabular}{l l p{10.5cm}}
        \hline
                              &        &                        \\
        \textbf{Abbreviation} & \vline & \textbf{Term}          \\
                              &        &                        \\\hline & & \\
        \textbf{R}\(_i\)      & \vline & i-th Requirement       \\
                              &        &                        \\\hline & & \\
        \textbf{i.e.}         & \vline & in other words         \\
                              &        &                        \\\hline & & \\
        \textbf{e.g.}         & \vline & for example            \\
                              &        &                        \\\hline & & \\
        \textbf{iff}          & \vline & if and only if         \\
                              &        &                        \\
        \hline
    \end{tabular}
    \caption{Abbreviations}
\end{table}

\section{Reference Documents}
\begin{itemize}
    \item CodeKataBattle RASD\footnote{\url{https://github.com/pontig/sw-eng-2-PasiniPontiggiaStasi/blob/main/RASD/RASD.pdf}}
    \item Assignment RDD A.Y. 2023-2024\footnote{\url{https://webeep.polimi.it/mod/folder/view.php?id=219353}}
    \item Course slides on WeeBeep\footnote{\url{https://webeep.polimi.it/mod/folder/view.php?id=207692}}
    \item ISO/IEC/IEEE 29148 dated 2018, \\
          Systems and software engineering - Life cycle processes - Requirements engineering\footnote{\url{https://www.iso.org/obp/ui/\#iso:std:iso-iec-ieee:29148:ed-2:v1:en}}
\end{itemize}

\newpage

\section{Document Structure}
The structure of this DD document follows six main sections:
\begin{enumerate}
    \item \textbf{Introduction:}
          provides a general overview of the system, its purpose, and its main components. 
          It describes the design as a whole, highlighting key terms, and references to related documents, and giving an idea of the overall design.

    \item \textbf{Architectural Design:}
          provides a high-level overview of how system responsibilities are distributed and assigned to subsystems. 
          It identifies each high-level subsystem and the roles assigned to them. 
          Additionally, it describes how these subsystems collaborate to achieve the desired functionality.
          
    \item \textbf{User Interface Design:}
          illustrates the design of the UI of the system, including UX flowcharts.
    
    \item \textbf{Requirements Traceability:}
          provides a cross-reference that connects system components to the requirements specified in the Requirements Analysis and Specification Document (RASD). 
          This cross-reference is presented in a tabular format, which clearly shows which system components fulfill each of the functional requirements defined in the RASD.

    \item \textbf{Implementation, Integration, and Test Plan:}   
          details the strategy and sequence for deploying subsystems and components. 
          It identifies the order in which sub-components are implemented, integrated, and tested. Additionally, it provides a comprehensive explanation of the methodologies and technologies to employ throughout the process.

    \item \textbf{Effort Spent:}
          keeps track of the time spent to complete this document.
          The first table defines the amount of hours used by the whole team to make important decisions and to make reviews,
          the other tables contain the individual effort spent by each team member.
\end{enumerate}