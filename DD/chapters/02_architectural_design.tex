\chapter{Architectural Design}
This section provides a top-down exploration of the system's architecture. 
It starts with an overview of the overall architecture, accompanied by diagrams illustrating its components. 
Subsequently, it utilizes an ER diagram to elucidate the system's data structure and a deployment view to outline the layers and tiers involved. 
Sequence diagrams are employed to illustrate the main runtime perspectives, while the class diagram delineates component interfaces. 
The section concludes by discussing the architectural design choices and their underlying rationales.

\section{Overview}
The CKB platform software architecture is organized into four main logical layers, which are associated with an autonomous tier so that the system fits in a 4-tier architecture:
\begin{itemize}
    \item the presentation tier, defines the user interface.
    \item the web server tier, in which the web server runs.
    \item the application tier, where data are processed.
    \item the data tier, where data are stored and managed.
\end{itemize}

\begin{figure}[H]
    \centering
    \includegraphics[width=\textwidth]{images/diagrams/high_level_diagram.png}
    \caption{High level components diagram}
\end{figure}

The system is accessible through a web interface and is developed as SPA. 
This web application type is ideal, as it facilitates extensive interaction without requiring frequent page reloads. 
In addition, this type of system allows for easy expansion of the services offered without major modifications to the system itself. 

The system architecture is organized into separate layers, where application servers communicate with a database management system and employ APIs for data retrieval and storage. 
Following REST standards, the application servers are deliberately designed as stateless, managing user login sessions through caching and adhering to web application best practices. 

Moreover, the system will incorporate multiple firewalls to enhance security.

\section{Component view}

The system is composed of the following components:

\begin{figure}[H]
    \centering
    \includegraphics[width=\textwidth]{images/diagrams/component_diagram.png}
    \caption{Component diagram}
\end{figure}

To maintain the readability of this diagram, the interfaces have been grouped according to their functionality.
A complete set of endpoints is available later in the document.

\subsection{Client components}
The client components are represented only by a single one, a user-friendly WebApp, that behaves differently depending on the type of user using it, an EDU or a STU. \\
The only logic incorporated is the ability to perform basic checks, such as the detection of incomplete forms or wrong data format. 
The WebApp is interfaced with the server components through all the APIs offered by the server, a detailed description is in the following sections.
\subsection{Server components}

\subsubsection*{Query manager}
The query manager is the component that handles the queries made by the other components that need to access the database. \\
It is responsible for the execution of the queries and for the communication with the database. \\
It is interfaced with all the internal models of the system that need to access the database, i.e. all the other components of the system except for the notification manager. \\
It is interfaced with the database through the DBMS API, external to the system.

\subsubsection*{Auth manager}
The auth manager is the component that handles the authentication of the users and the authorization of the requests made by the other components that need to access the database concerning the user who made the request.\\
It is interfaced with all the internal models of the system that behave differently depending on the level of the user that made the request, i.e. the badges manager, the tournament manager, and the battle manager.\\
It isn't interfaced with any external component.

\subsubsection*{Notification manager}
The notification manager is the component that handles the need of the system to notify the users of some events, such as the start of a tournament or the end of a battle. \\
It is interfaced with all the internal models of the system that need to notify the users, i.e. the tournament manager and the battle manager. \\
It is interfaced with the Email API, external to the system.

\subsubsection*{Badges manager}
The badges manager is the component that handles the gamification badges. \\
It allows:
\begin{itemize}
    \item the creation of new badges;
    \item the assignment of badges to the STUs;
\end{itemize}
It is interfaced with the auth manager since the creation of a badge is admissible only by the EDUs. \\
It is interfaced with the query manager to access the database to store the badges. \\
It is interfaced with the EDU WebApp through the proper APIs, external to the system.

\subsubsection*{Tournaments manager}
The tournament manager is the component that handles the management of the tournaments.\\
It allows:
\begin{itemize}
    \item the creation of new tournaments;
    \item the closure of the tournaments;
    \item the visualization of the tournaments;
    \item the exchange of admin permission between EDUs;
    \item the subscription of the STUs to the tournaments;
    \item the visualization of the STUs' score in the tournaments, and so the overall ranking;
    \item the visualization of the battles within a tournament.
\end{itemize}
It is interfaced with the auth manager to allow and perform different actions depending on the level of the user that made the request. \\
It is interfaced with the query manager to access the database to store or retrieve data. \\
It is interfaced with the notification manager to notify the users about the start and the end of a tournament.\\
It is interfaced with the WebApp, both EDU and STU, through the proper APIs, external to the system.

\subsubsection*{Battles manager}
The battles manager is the component that handles the management of the battles.\\
It allows:
\begin{itemize}
    \item the creation of new battles;
    \item the subscription of the STUs to the battles;
    \item the visualization of the scores of the teams in the battles, and so of the ranking
    \item the visualization of the battles;
    \item the automatic evaluation of the battles;
    \item the eventual manual evaluation of the battles by the EDUs
\end{itemize}
It is interfaced with the auth manager to allow and perform different actions depending on the level of the user that made the request. \\
It is interfaced with the query manager to access the database to store or retrieve data. \\
It is interfaced with the notification manager to notify the users of the opening of subscriptions, the start and the end of a battle.\\
It is interfaced with the WebApp, both EDU and STU and the proper APIs and with GitHub (to perform the manual evaluation), all external to the system. \\
It is interfaced with the badges manager in order to check if a badge has been achieved or not.

\subsubsection*{Profile Inspector}
The profile inspector is the component that handles the visualization of the profiles of the STUs and, consequently, of the badges that it has earned during the tournaments. \\
It is interfaced with the query manager to access the database to retrieve data. \\
It is interfaced with the WebApp, both EDU and STU, through the proper APIs, external to the system.

\subsection{Logical description of the data}
The data of the system is organized in a relational database, following the below ER diagram:

\begin{figure}[H]
    \centering
    \includegraphics[width=0.75\textwidth]{images/diagrams/er_diagram.png}
    \caption{Entity-relationship diagram}
    \label{fig:er_diagram}
\end{figure}

\section{Deployment view}
The diagram \ref{fig:deployment_diagram} shows the deployment view of the system.\\
Our system comprises two essential components: a static web server and an application server. 
The static web server serves as the entry point for clients to access the SPA, while the application server furnishes the necessary APIs for the SPA's functionality. 
To optimize performance, these components follow distinct solutions.\\
The static web server is hosted on a CDN (Content Delivery Network) on the cloud, exploiting its edge location caches and reverse proxies to ensure rapid response times. \\
On the other hand, the application server, containing both the business logic and the data layer, is home-based on a cloud provider. \\
This decision offers numerous advantages over traditional in-house hosting, including:
\begin{itemize}
    \item \textbf{Scalability and Flexibility}: The cloud infrastructure allows for the dynamic addition or removal of resources like virtual machines, performance cores, or memory as per the evolving needs. Load balancing services further enable the application server to adapt seamlessly to changes in traffic or workload.
    \item \textbf{Security}: Enhanced security features, such as live monitoring and firewalls, contribute to safeguarding the application server against potential data breaches, cyberattacks, and other security threats.
    \item \textbf{Cost-efficiency}: The cloud provider's pay-as-you-go model ensures cost efficiency by charging only for the utilized resources. This approach helps in reducing overall costs, making it a financially prudent choice.
\end{itemize}
These attributes position a cloud provider as an ideal hosting solution for large, high-traffic applications. 
The selected cloud provider must respect all these features to effectively meet our system requirements.

\begin{figure}[ht]
    \centering
    \includegraphics[width=\textwidth]{images/diagrams/deployment_view.png}
    \caption{Deployment diagram}
    \label{fig:deployment_diagram}
\end{figure}

The components of the system are explained in the following:
\begin{itemize}
    \item \textbf{PC}: Personal computer of the user, it suffices to have a working OS and a Browser installed that supports JavaScript and HTML5 to use the system.
    \item \textbf{CDN}: The CDN is used to host the static web server, that serves as the entry point for clients to access the SPA. 
                        It allows to download the static web pages without affecting the performance of the main application server. 
                        The SPA is static and all of its code is run on the client's machine, so there is no need for any logic to be implemented on the CDN side.
    \item \textbf{Cloud provider}: The cloud provider is used to host the application server. 
                                   It allows the system to be scalable, flexible, secure, and cost-efficient.\\
          It is composed by:
          \begin{itemize}
              \item \textbf{Load balancer}: The load balancer is used to distribute the traffic between the different instances of the application server. 
                                            It is used to make the system scalable and flexible.
              \item \textbf{Application servers}: The application servers are used to host the application server. 
                                                  They are in an array of instances so that the load balancer can distribute the traffic between them. 
                                                  The number of instances can be changed dynamically so that the system can adapt to the traffic. 
                                                  They are used to make the system scalable and flexible.
              \item \textbf{DBMS server}: The DBMS server is used to host the database.
              \item \textbf{Firewalls}: Firewalls are used to make the system secure, and are placed between the load balancer and the external world and between the application servers and the DBMS server. 
                                        They provide an additional layer of security by blocking or allowing traffic based on predetermined rules. 
                                        This helps to protect the system from unauthorized access or malicious attacks.
          \end{itemize}
\end{itemize}

\section{Runtime View}
This section describes the most important components interactions of the system.

\subsection{User login}
At the beginning the end user must log in to use the main functions of the application. The login is done by enetring the email and password. If the credentials are present in the database and correct the process will be successful and it will be able to open its dashboard, otherwise it will have to repeat the procedure. (Figure \ref{fig:RuntimeView_UserLogin})
\begin{figure}[H]
    \centering
    \includegraphics[width=\textwidth]{images/runtimeviews/RuntimeView_UserLogin.png}
    \caption{Runtime View of User Login Event}
    \label{fig:RuntimeView_UserLogin}
\end{figure}

\subsection{Create a tournament}
The following sequence diagram is used to explain how to create a tournament. The end user from its device can create a tournament entering the correlated details through the related component, which are sent to the database. If the tournament has been created successfully, the correlated page is created.(Figure \ref{fig:RuntimeView_CreateTournament} )
\begin{figure}[H]
    \centering
    \includegraphics[width=\textwidth]{images/runtimeviews/RuntimeView_CreateTournament.png}
    \caption{Runtime View of Create a Tournament Event}
    \label{fig:RuntimeView_CreateTournament}
\end{figure}

\subsection{Create a battle}
The following sequence diagram is used to explain how to create a battle within a tournament. The end user from its device can create a battle entering the correlated details through the related component. If the user is a granted educator for the current tournament, then the details of the new battle are sent to the database and if the battle has been created successfully, the correlated page is created and all the users subscribed to the correlated tournament are notified. (Figure \ref{fig:RuntimeView_CreateBattle})
\begin{figure}[H]
    \centering
    \includegraphics[width=\textwidth]{images/runtimeviews/RuntimeView_CreateBattle.png}
    \caption{Runtime View of Create a Battle Event}
    \label{fig:RuntimeView_CreateBattle}
\end{figure}

\subsection{Join a Battle Solo}
In this sequence diagram is shown how an user can subscribe to a battle. By joining the battle, the system add to the database the email of the user into the subscribed email of that battle. If the user decides to invite other members, the inserted emails are notified. (Figure \ref{fig:RuntimeView_JoinBattleSolo})
\begin{figure}[H]
    \centering
    \includegraphics[width=\textwidth]{images/runtimeviews/RuntimeView_JoinBattleSolo.png}
    \caption{Runtime View of joining a Battle Solo}
    \label{fig:RuntimeView_JoinBattleSolo}
\end{figure}

\subsection{Upload code}
In this case, the GitHub API notify the system which through the related component computes the new score and update it by sending the new score to the database. Finally, the system updates the battle ranking with the new score. (Figure \ref{fig:RuntimeView_CodeUploaded})
\begin{figure}[H]
    \centering
    \includegraphics[width=\textwidth]{images/runtimeviews/RuntimeView_CodeUploaded.png}
    \caption{Runtime View of Notification of a new commit of a team}
    \label{fig:RuntimeView_CodeUploaded}
\end{figure}

\section{API endpoints}
% Define JavaScript language style
\lstdefinelanguage{JavaScript}{
    keywords={const, let, var, if, else, for, while, function},
    keywordstyle=\color{blue},
    commentstyle=\color{gray},
    stringstyle=\color{ForestGreen},
    morecomment=[l]{//},
    morecomment=[s]{/*}{*/},
    morestring=[b]",
    morestring=[b]'
}

% Set general listings style
\lstset{
    basicstyle=\ttfamily\footnotesize,
    showstringspaces=false,
    tabsize=2,
    breaklines=true
}

The following section describes one by one the endpoints of the APIs offered by the system.\\
For the "battle" object, the "phase" attribute is intended to be as an enumeration of 4 possible values:
\begin{itemize}
    \item 1 - Registration
    \item 2 - Code Submission
    \item 3 - Manual evaluation
    \item 4 - Finished, closed
\end{itemize}

\section*{GET endpoints - General User}
In these endpoints it is not required to pass the id of the user actually connected.
\subsection*{Search for a student}
This endpoint allows the anyone who writes in the searchbox to search for a STU by name, surname or email.\\
\textbf{Method} : GET \\
\textbf{EndPoint} : /students/:\{query\} \\
\textbf{URL Parameters} :
\begin{itemize}
    \item {\ttfamily{query}} : the string that the user has written in the searchbox\\
\end{itemize}
\textbf{Response} :
\begin{itemize}
    \item {\ttfamily{200}} : JSON object
          \begin{lstlisting}[language=JavaScript, label={lst:jscode}, basicstyle=\ttfamily]
[
    {
        "id": Number,
        "name": String,
        "surname": String,
    },
    ...
]
            \end{lstlisting}
    \item {\ttfamily{400}} : error String("The searchbox is empty, at least 1 character is required")
    \item {\ttfamily{401}} : error String("Unauthorized")
\end{itemize}


\subsection*{Search for an EDU}
This endpoint allows to search for an EDU by name, surname or email\\
\textbf{Method} : GET \\
\textbf{EndPoint} : /educators/\{query\}  \\
\textbf{URL Parameters} :
\begin{itemize}
    \item {\ttfamily{query}} : the string that the user has written in the searchbox\\
\end{itemize}
\textbf{Response} :
\begin{itemize}
    \item {\ttfamily{200}} : JSON object
          \begin{lstlisting}[language=JavaScript, label={lst:jscode}, basicstyle=\ttfamily]
[
    {
        "id": Number,
        "name": String,
        "surname": String,
    },
    ...
]
            \end{lstlisting}
    \item {\ttfamily{400}} : error String("The searchbox is empty, at least 1 character is required")
    \item {\ttfamily{401}} : error String("Unauthorized")
\end{itemize}

\subsection*{Inspect a STU's profile}
This endpoint allows the anyone who clicks on a STU's name in the search results to inspect its profile.\\ % or click on a STU's name in any point of the system
\textbf{Method} : GET \\
\textbf{EndPoint} : /student/\{id\}/profile \\
\textbf{URL Parameters} :
    \begin{itemize}
        \item {\ttfamily{id}} : the id of the student the user is looking for
    \end{itemize}
\textbf{Response} : 
    \begin{itemize}
        \item {\ttfamily{200}} : JSON object
        \begin{lstlisting}[language=JavaScript, label={lst:jscode}, basicstyle=\ttfamily]
{
    "name": String,
    "surname": String,  
    "tournaments": [
        {
            "id": Number,
            "name": String
        },
        ...
    ],
    "badges": [
        {
            "id": Number,
            "name": String,
            "rank": Number[1 - 7],
            "obtained": [
                {
                    "date": String(Date),
                    "tournament": String,
                    "tournament_id": Number,
                    "battle": String,
                    "battle_id": Number
                },
                ...
            ]
        },
        ...
    ]
}
        \end{lstlisting}
        \item {\ttfamily{401}} : error String("unauthorized")
    \end{itemize}



\section*{GET endpoints - EDU}
In these endpoints each request will have to pass also a specific field in the body.\\ 
This field is the id of the user actually connected to the platform and who is sending the request in order to check if the user is of the correct role.

\subsection*{Get all tournaments}
This endpoint allows the EDU to get the list of all the tournaments on the CKB platform, it will be accessed whenever the EDU clicks on the "Show all tournaments" button on the dashboard.\\
\textbf{Method} : GET \\
\textbf{EndPoint} : /tournaments \\
\textbf{URL Parameters}: None \\
\textbf{Response}:
\begin{itemize}
    \item {\ttfamily{200}} : JSON object
          \begin{lstlisting}[language=JavaScript, label={lst:jscode}, basicstyle=\ttfamily]
[
    {
        "id": Number,
        //name of the tournament
        "name": String,
        // first name of the tournament creator
        "first_name": String, 
        // last name of the tournament creator
        "last_name": String,
        // if the tournament is open (true) or closed (false)
        "active": Boolean,
    },
    ...
]
        \end{lstlisting}
    \item {\ttfamily{401}}: error String("unauthorized)
\end{itemize}

\subsection*{Get owned tournaments}
This endpoint allows the EDU to get the list of the tournaments that it owns, it will be accessed every time the EDU opens the dashboard.\\

\textbf{Method} : GET \\
\textbf{EndPoint} : /tournaments/owned/\{edu\_id\} \\
\textbf{URL Parameters}:
\begin{itemize}
    \item {\ttfamily{edu\_id}}: the id of the EDU that is requesting the list of the tournaments
\end{itemize}
\textbf{Response}:
\begin{itemize}
    \item {\ttfamily{200}} : JSON object
          \begin{lstlisting}[language=JavaScript, label={lst:jscode}, basicstyle=\ttfamily]
[
    {
        "id": Number,
        //name of the tournament
        "name": String,
        // first name of the tournament creator
        "first_name": String, 
        // last name of the tournament creator
        "last_name": String,
        // if the tournament is open (true) or closed (false)
        "active": Boolean,
    },
    ...
]
            \end{lstlisting}
    \item {\ttfamily{401}} : error String("Unauthorized")
\end{itemize}

\subsection*{Get tournament details}
This endpoint allows the EDU to get the details of a tournament, it will be accessed when an EDU selects a tournament from the list of the tournaments in the dashboard.\\
\textbf{Method} : GET \\
\textbf{EndPoint} : /tournaments/\{id\} \\
\textbf{URL Parameters}:
\begin{itemize}
    \item {\ttfamily{id}} : id of the tournament
\end{itemize}
\textbf{Body Parameters} :
\begin{itemize}
    \item {\ttfamily{edu\_id}}: the id of the EDU that is requesting the details of the tournament
\end{itemize}
\textbf{Response}:
\begin{itemize}
    \item {\ttfamily{200}} : JSON object
          \begin{lstlisting}[language=JavaScript, label={lst:jscode}, basicstyle=\ttfamily]
[
    {
        "id": Number,
        "name": String,
        "active": Boolean,
        "admin": Boolean,
        "battles": [
            {
                "id": Number,
                "name": String,
                "language": String,
                "participants": Number,
                "phase": Number[1 - 4],
                // Time remaining in the current phase
                "remaining": String(Date)
            },
            ...
        ],
        "ranking": [
            {
                "id": Number,
                "name": String,
                "points": Number
            },
            ...
        ],
    }
]
            \end{lstlisting}
    \item {\ttfamily{401}} : error String("Unauthorized")
\end{itemize}

\subsection*{Get battle details}
This endpoint allows the EDU to get the details of a battle, it will be accessed when an EDU selects a battle from the list of the battles in a tournament view.\\

\textbf{Method} : GET \\
\textbf{EndPoint} : /battles/\{id\}\\
\textbf{URL Parameters}:
\begin{itemize}
    \item {\ttfamily{id}}: the id of the battle that the EDU wants to get the details of
\end{itemize}
\textbf{Response}:
\begin{itemize}
    \item {\ttfamily{200}} : JSON object
          \begin{lstlisting}[language=JavaScript, label={lst:jscode}, basicstyle=\ttfamily]
[
    {
        "id": Number,
        "title": String,
        "description": String,
        "language": String,
        "opening": String(Date),
        "registration": String(Date),
        "closing": String(Date),
        "min_group_size": Number,
        "max_group_size": Number,
        "link": String,
        "phase": Number[1 - 4],
        "tournament_name": String,
        "tournament_id": Number,
        "admin": Boolean,
        // Could be reduntant with phase, but it's easier to check
        "manual": Boolean,
        "ranking": [
            {
                "id": Number,
                "name": String,
                "score": Number
            },
            ...
        ],
    }
]
            \end{lstlisting}
    \item {\ttfamily{401}} : error String("Unauthorized")
\end{itemize}

\subsection*{Get the list of groups for the manual evaluation}
This endpoint allows the EDU to get the list of the groups subscribed to a battle in the "manual evaluation" stage, and for each group that has already been evaluated it provides the score that it has obtained.\\
\textbf{Method} : GET \\
\textbf{EndPoint} : /battles/:battle\_id/manualEvaluation \\
\textbf{URL Parameters}:
\begin{itemize}
    \item {\ttfamily{tournament\_id}} : the id of the tournament of the correlated battle
    \item {\ttfamily{battle\_id}} : the id of the battle that the EDU wants to get the list of the codes to evaluate of
    \item {\ttfamily{manualEvaluation}} : it is a parameter that represent the boolean variable "manual" of the battle
    \item {\ttfamily{phase}} : it is a parameter that represents the phase in which a specific battle is. In this case, the battle must be in phase 3, which corresponds to the manual evalation
\end{itemize}
\textbf{Response}:
\begin{itemize}
    \item {\ttfamily{200}} : JSON object
          \begin{lstlisting}[language=JavaScript, label={lst:jscode}, basicstyle=\ttfamily]
[
    {
        "id": Number,
        "team": String,
        "score"?: Number
    },
    ...
]
        \end{lstlisting}
    \item {\ttfamily{401}} : String("unauthorized")
\end{itemize}

\subsection*{Evaluate code}
This endpoint allows the EDU to evaluate a code, it will be accessed when an EDU clicks on the "Evaluate" button from the table of the codes to evaluate.\\
\textbf{Method} : GET \\
\textbf{EndPoint} : /battles/\{battle\_id\}/teams/\{team\_id\}     \\
\textbf{URL Parameters}:
\begin{itemize}
    \item {\ttfamily{tournament\_id}} : the id of the tournament of the correlated battle
    \item {\ttfamily{team\_id}}: the id of the group that the EDU wants to evaluate
    \item {\ttfamily{battle\_id}}: the id of the battle that the EDU wants to evaluate
\end{itemize}
\textbf{Response}:
\begin{itemize}
    \item {\ttfamily{200}} : JSON object
          \begin{lstlisting}[language=JavaScript, label={lst:jscode}, basicstyle=\ttfamily]
{
    "group_id": Number,
    "battle_id": Number,
    "language": String,
    // Name of the group
    "name": String,
    // Score of the group, null if not evaluated yet
    "score"?: Number,
    // Code of the group. The url links to the file
    "code": String(url)
}
    \end{lstlisting}
    \item {\ttfamily{401}} : error String("unauthorized")
\end{itemize}


\section*{GET endpoints - STU}
In these endpoints each request will have to pass also a specific field in the body.\\ 
This field is the id of the user actually connected to the platform and who is sending the request in order to check if the user is of the correct role.

\subsection*{Get subscribed tournaments}
This endpoint allows the STU to get the list of the tournaments that it is subscribed to, it will be accessed every time the STU opens the dashboard.\\
\textbf{Method} : GET \\
\textbf{EndPoint} : /tournaments/subscribed     \\
\textbf{Response}:
    \begin{itemize}
        \item {\ttfamily{200}} : JSON object 
        \begin{lstlisting}[language=JavaScript, label={lst:jscode}, basicstyle=\ttfamily]
[
    {
        "id": Number,
        "name": String,
        // first name of the tournament creator
        "first_name": String,
        // last name of the tournament creator
        "last_name": String,
        // if the tournament is open (true) or closed (false)
        "active": Boolean,
    },
    ...
]
            \end{lstlisting}
        \item {\ttfamily{401}} : error String("unauthorized") 
    \end{itemize}

\subsection*{Get unsubscribed tournaments}
This endpoint allows the STU to get the list of the tournaments that it is not subscribed to, it will be accessed every time the STU opens the dashboard.\\
\textbf{Method} : GET \\
\textbf{EndPoint} : /tournament/unsubscribed     \\
\textbf{Response}:
    \begin{itemize}
        \item {\ttfamily{200}} : JSON object
        \begin{lstlisting}[language=JavaScript, label={lst:jscode}, basicstyle=\ttfamily]
[
    {
        "id": Number,
        "name": String,
        "daysLeft": Number,
    },
    ...
]
        \end{lstlisting}
        \item {\ttfamily{401}} : error String("unauthorized") 
        \end{itemize}

\subsection*{Get tournament details}
This endpoint allows the STU to get the details of a tournament, it will be accessed when a STU selects a tournament from the list of the tournaments in the dashboard.\\
\textbf{Method} : GET \\
\textbf{EndPoint} : /tournaments/\{t\_id\}\\
\textbf{URL Parameters}:
        \begin{itemize}
            \item {\ttfamily{t\_id}} : this is the id of the tournament the user is looking for
        \end{itemize}
\textbf{Response}:
        \begin{itemize}
            \item {\ttfamily{200}} : JSON object
            \begin{lstlisting}[language=JavaScript, label={lst:jscode}, basicstyle=\ttfamily]
[
    {
        "id": Number,
        "name": String,
        "active": Boolean,
        "canSubscribe": Boolean,
        "subscribed": Boolean,
        "battles": [
            {
                "id": Number,
                "name": String,
                "language": String,
                "participants": Number,
                "subscribed": Boolean,
                "score"?: Number,
                "phase": Number[1 - 4],
                // Time remaining in the current phase
                "remaining": String(Date)
            },
            ...
        ],
        "ranking": [
            {
                "id": Number,
                "name": String,
                "points": Number
            },
            ...
        ]
    }
]
            \end{lstlisting}
            \item {\ttfamily{401}} : error String("unauthorized")
        \end{itemize}

\subsection*{Get battle details}
This endpoint allows the STU to get the details of a battle, it will be accessed when a STU selects a battle from the list of the battles in a tournament view.\\
\textbf{Method} : GET \\
\textbf{EndPoint} :  battles/\{battle\_id\}\\
\textbf{URL Parameters}:
\begin{itemize}
    \item {\ttfamily{battle\_id}} : this is the id of the battle the user is looking for
\end{itemize}
\textbf{Response} : 
    \begin{itemize}
        \item {\ttfamily{200}} : JSON object
        \begin{lstlisting}[language=JavaScript, label={lst:jscode}, basicstyle=\ttfamily]
[
    {
        "id": Number,
        "title": String,
        "description": String,
        "language": String,
        "opening": String(Date),
        "registration": String(Date),
        "closing": String(Date),
        "min_group_size": Number,
        "max_group_size": Number,
        "link": String,
        "canSubscribe": Boolean,
        "canInviteOthers": Boolean,
        // If the user's team satisfies the min members constraint
        "minConstraintSatisfied": Boolean,
        "subscribed": Boolean,
        "phase": Number[1 - 4],
        "tournament_name": String,
        "tournament_id": Number,
        "ranking": [
            {
                "id": Number,
                "name": String,
                "score": Number
            },
            ...
        ]
    }
]
        \end{lstlisting} 
        \item {\ttfamily{401}} : error String("unauthorized")
    \end{itemize}

\section*{POST endpoints - General User}
\subsection*{Registration}
This endpoint allows the user to register to the platform.\\
\textbf{Method} : POST \\
\textbf{EndPoint} : {\color{blue} TODO}
\textbf{Body Parameters} :
\begin{itemize}
    \item {\ttfamily{name}} : the name of the user
    \item {\ttfamily{surname}} : the surname of the user
    \item {\ttfamily{email}} : the email of the user
    \item {\ttfamily{uni}} : the university of the user
    \item {\ttfamily{role}} : the role of the user (with value "STU" or "EDU")
    \item {\ttfamily{password}} : the password of the user
    \item {\ttfamily{password2}} : the password of the user, repeated
\end{itemize}
\textbf{Response} :
\begin{itemize}
    \item {\ttfamily{200}} : message String("")
    \item {\ttfamily{400}} : error String("")
    \item {\ttfamily{409}} : error String("")
\end{itemize}

\subsection*{Login}
This endpoint allows the user to login to the platform.\\
\textbf{Method} : POST \\
\textbf{EndPoint} : {\color{blue} TODO}
\textbf{Body Parameters} :
\begin{itemize}
    \item {\ttfamily{email}} : the email of the user
    \item {\ttfamily{password}} : the password of the user
\end{itemize}
\textbf{Response} :
\begin{itemize}
    \item {\ttfamily{200}} : message String("")
    \item {\ttfamily{400}} : error String("")
    \item {\ttfamily{401}} : error String("")
\end{itemize}

\section*{POST endpoints - EDU}
\subsection*{Create Tournament}
This endpoint allows the EDU to create a new tournament.\\
\textbf{Method} : POST \\
\textbf{EndPoint} : {\color{blue} TODO}
\textbf{URL Parameters}:
\begin{itemize}
    \item {\ttfamily{EDU\_id}} : Integer
\end{itemize}
\textbf{Body Parameters} :
\begin{itemize}
    \item {\ttfamily{tournemant\_name}} : String
    \item {\ttfamily{registration\_deadline}} : String (date)
    \item {\ttfamily{badges}} : Array of Objects (badge\_id)
\end{itemize}
\textbf{Response} :
\begin{itemize}
    \item {\ttfamily{200}} : message String("")
    \item {\ttfamily{400}} : error String("")
    \item {\ttfamily{403}} : error String("")
    \item {\ttfamily{404}} : error String("")
\end{itemize}

\subsection*{Share Tournament}
This endpoint allows the EDU to share a tournament with other colleagues.\\
\textbf{Method} : POST \\
\textbf{EndPoint} : {\color{blue} TODO}
\textbf{URL Parameters}:
\begin{itemize}
    \item {\ttfamily{EDU\_id}} : Integer
    \item {\ttfamily{tournament\_id}} : Integer
\end{itemize}
\textbf{Body Parameters} :
\begin{itemize}
    \item {\ttfamily{invited\_EDU\_id}} : Integer
\end{itemize}
\textbf{Response} :
\begin{itemize}
    \item {\ttfamily{200}} : message String("")
    \item {\ttfamily{400}} : error String("")
    \item {\ttfamily{403}} : error String("")
    \item {\ttfamily{404}} : error String("")
\end{itemize}

\subsection*{Create Battle}
This endpoint allows the EDU to create a new battle within the context of a tournament.\\
\textbf{Method} : POST \\
\textbf{EndPoint} : {\color{blue} TODO}
\textbf{URL Parameters}:
\begin{itemize}
    \item {\ttfamily{EDU\_id}} : Integer
    \item {\ttfamily{tournament\_id}} : Integer
\end{itemize}
\textbf{Body Parameters} :
\begin{itemize}
    \item {\ttfamily{battle\_name}} : String
    \item {\ttfamily{programming\_language}} : String
    \item {\ttfamily{registration\_deadline}} : String (date)
    \item {\ttfamily{end\_battle}} : String (date)
    \item {\ttfamily{manual\_evaluation}} : Boolean
    \item {\ttfamily{min\_STU}} : Integer
    \item {\ttfamily{max\_STU}} : Integer
    
    {\color{red}
    Gradle => Da modificare con footer
    \item {\ttfamily{test\_cases}} : ?
    \item {\ttfamily{build\_script}} : ?
    \item {\ttfamily{problem\_spec}} : File PDF
    }
    
\end{itemize}
\textbf{Response} :
\begin{itemize}
    \item {\ttfamily{200}} : message String("")
    \item {\ttfamily{400}} : error String("")
    \item {\ttfamily{403}} : error String("")
    \item {\ttfamily{404}} : error String("")
\end{itemize}

\subsection*{Create Badge}
This endpoint allows the EDU to create a new badge insisde the CKB platform.\\
\textbf{Method} : POST \\
\textbf{EndPoint} : {\color{blue} TODO}
\textbf{URL Parameters}:
\begin{itemize}
    \item {\ttfamily{EDU\_id}} : Integer
\end{itemize}
\textbf{Body Parameters} :
\begin{itemize}
    \item {\ttfamily{badge\_name}} : String
    \item {\ttfamily{rules}} : Array of String    
\end{itemize}
\textbf{Response} :
\begin{itemize}
    \item {\ttfamily{200}} : message String("")
    \item {\ttfamily{400}} : error String("")
    \item {\ttfamily{403}} : error String("")
    \item {\ttfamily{404}} : error String("")
\end{itemize}

\subsection*{Manual Evaluation}
This endpoint allows the EDU to evaluate perform a manual evaluation on team's code.\\
\textbf{Method} : POST \\
\textbf{EndPoint} : {\color{blue} TODO}
\textbf{URL Parameters}:
\begin{itemize}
    \item {\ttfamily{EDU\_id}} : Integer
    \item {\ttfamily{tournament\_id}} : Integer
    \item {\ttfamily{battle\_id}} : Integer
\end{itemize}
\textbf{Body Parameters} :
\begin{itemize}
    \item {\ttfamily{given\_grades}} : Array of Tuple (team\_id - score)   
\end{itemize}
\textbf{Response} :
\begin{itemize}
    \item {\ttfamily{200}} : message String("")
    \item {\ttfamily{400}} : error String("")
    \item {\ttfamily{403}} : error String("")
    \item {\ttfamily{404}} : error String("")
\end{itemize}

\section*{POST endpoints - STU}
\subsection*{Join Tournament}
This endpoint allows the STU to join a tournament.\\
\textbf{Method} : POST \\
\textbf{EndPoint} : {\color{blue} TODO}
\textbf{URL Parameters}:
\begin{itemize}
    \item {\ttfamily{STU\_id}} : Integer
\end{itemize}
\textbf{Body Parameters} :
\begin{itemize}
    \item {\ttfamily{tournament\_id}} : Integer   
\end{itemize}
\textbf{Response} :
\begin{itemize}
    \item {\ttfamily{200}} : message String("")
    \item {\ttfamily{400}} : error String("")
    \item {\ttfamily{403}} : error String("")
    \item {\ttfamily{404}} : error String("")
    \item {\ttfamily{409}} : error String("")
\end{itemize}

\subsection*{Join Battle}
This endpoint allows the STU to join a battle forming a team respection the boundaries imposed.\\
\textbf{Method} : POST \\
\textbf{EndPoint} : {\color{blue} TODO}
\textbf{URL Parameters}:
\begin{itemize}
    \item {\ttfamily{STU\_id}} : Integer
    \item {\ttfamily{tournament\_id}} : Integer   
\end{itemize}
\textbf{Body Parameters} :
\begin{itemize}
    \item {\ttfamily{battle\_id}} : Integer  
    \item {\ttfamily{team\_name}} : String  
    \item {\ttfamily{STU\_invited}} : Array of STU (STU\_id)  
\end{itemize}
\textbf{Response} :
\begin{itemize}
    \item {\ttfamily{200}} : message String("")
    \item {\ttfamily{400}} : error String("")
    \item {\ttfamily{403}} : error String("")
    \item {\ttfamily{404}} : error String("")
    \item {\ttfamily{409}} : error String("")
\end{itemize}

\subsection*{Join Team}
This endpoint allows the STU to join a team to compete in a battle.\\
\textbf{Method} : POST \\
\textbf{EndPoint} : {\color{blue} TODO}
\textbf{URL Parameters}:
\begin{itemize}
    \item {\ttfamily{STU\_id}} : Integer
\end{itemize}
\textbf{Body Parameters} :
\begin{itemize}
    \item {\ttfamily{team\_id}} : String  
    \item {\ttfamily{tournament\_id}} : Integer  
    \item {\ttfamily{battle\_id}} : Integer  
    \item {\ttfamily{choice}} : Boolean 
\end{itemize}
\textbf{Response} :
\begin{itemize}
    \item {\ttfamily{200}} : message String("")
    \item {\ttfamily{400}} : error String("")
    \item {\ttfamily{403}} : error String("")
    \item {\ttfamily{404}} : error String("")
    \item {\ttfamily{409}} : error String("")
\end{itemize}

\newpage

\section{Component interfaces}
{\color{green} Da contorllare se i nomi dei metodi usati sono consistenti con quelli delle Runtime View \\}
In this diagram are displayed all the functions that each component uses to communicate with other components. In particular, the blue interfaces are external APIs in which at this level of abstraction do not require to be detailed as the other interfaces. The interfaces contains both the operations for STU users and EDU users. 
\begin{figure}[H]
    \centering
    \includegraphics[width=\textwidth]{images/diagrams/ComponentInterfaces.png}
    \caption{Component interfaces diagram}
\end{figure}

\newpage

\section{Selected Architectural Styles and Patterns}
\subsection{Client - Server Paradigm}
The client-server paradigm represents a software architecture model in which two distinct components, identified as the client and server, interact to deliver a service. 
The client is the component that requests the service, while the server is responsible for its provision.
Communication between them is based on the concept of asynchrony, allowing the client to send requests without waiting for an immediate response. 
The server, in turn, can process the request at a later time and send a response to the client. This highly flexible paradigm finds extensive use in implementing various applications, including web, network, and distributed contexts.
Key advantages include flexibility, scalability, and the ability to configure the system to ensure an adequate level of security.

\subsection{4 - tiers Architecture}
The 4-tier architecture is a software architecture model that divides an application into four distinct layers: Client, Web Server, Application Server, and Database Server. 
In its implementation for web applications, users send requests to the web server through the browser, which provides static content and forwards dynamic requests to the application server. 
The application server processes the requests, interacts with the database server, and sends responses to the web server, which, in turn, transmits them to the client.
The advantages of this architecture include scalability, enhanced security through data isolation, and increased ease of maintenance and testing due to the separation of various layers. 
The 4-tier architecture is particularly suitable for complex and high-traffic web applications, offering flexibility and scalability to meet a wide range of application needs.

\subsection{MVC Design Pattern}
The MVC is a widely adopted architectural pattern for the development of UI, separating the logic of the user interface from the business logic to enhance maintainability and scalability. 
Comprising three main components:
\begin{itemize}
    \item \textbf{Model:} manages data and business rules.
    \item \textbf{View:} displays data to the user.
    \item \textbf{Controller:} handles user requests.
\end{itemize} 
Advantages of the MVC pattern include the separation of concerns, ease of scalability, and maintainability of the application. 
MVC is a versatile model suitable for various applications, particularly for web applications.

\subsection{RESTful APIs}
The RESTful API architecture is a software model for creating APIs based on the HTTP protocol. 
RESTful APIs consist of resources identified by URLs and operate through HTTP methods such as GET, POST, PUT, and DELETE. 
They adhere to the principles of REST, ensuring flexibility, scalability, and ease of use. 
Key principles include the use of unique URIs for each resource, the adoption of HTTP methods to define operations, and the standardized representation of data (such as JSON or XML). 
RESTful APIs have been widely embraced for web API development due to their flexibility, scalability, and ease of implementation.

\section{Other design decisions}
\subsection{Web Application}
As the platform's primary functionality is closely tied to coding activities, it is expected that users will predominantly employ personal computers, so there is no need to develop a mobile application, which would require a significant amount of additional work. \\
Instead, the system will be accessible through a web application, that is much easier to develop, maintain, and access by the users.

\subsection{Single Page Application}
The system is developed as a SPA, which is a web application that interacts with the user by dynamically rewriting the current web page with new data from the web server, instead of the default method of the browser loading entire new pages.\\
This approach allows the system to be more responsive and similar to a desktop application since the user does not wait for the entire page to be reloaded every time it performs an action. 
Furthermore, it will allow the system to be more efficient, since the server does not have to send the entire page every time the user performs an action, but only the data that has changed, leaving the client to render the page.

\subsection{Relational Database}
The system is designed to use a relational database because it is effective at storing structured data, granting data integrity, and providing fast query performance. 
It can also be easily scaled to handle large amounts of data and support many concurrent users. 
The database allows us to store and retrieve information efficiently, while also ensuring that the data is accurate and consistent.

\subsection{- TODO -}
{\color{red}Non abbiamo mai parlato di Gradle secondo me nel design è da mettere, non so bene dove, ma bisonga mostrare che è parte del sistema \\}